\documentclass[12pt,]{book}
\usepackage{lmodern}
\usepackage{setspace}
\setstretch{1.5}
\usepackage{amssymb,amsmath}
\usepackage{ifxetex,ifluatex}
\usepackage{fixltx2e} % provides \textsubscript
\ifnum 0\ifxetex 1\fi\ifluatex 1\fi=0 % if pdftex
  \usepackage[T1]{fontenc}
  \usepackage[utf8]{inputenc}
\else % if luatex or xelatex
  \ifxetex
    \usepackage{mathspec}
  \else
    \usepackage{fontspec}
  \fi
  \defaultfontfeatures{Ligatures=TeX,Scale=MatchLowercase}
\fi
% use upquote if available, for straight quotes in verbatim environments
\IfFileExists{upquote.sty}{\usepackage{upquote}}{}
% use microtype if available
\IfFileExists{microtype.sty}{%
\usepackage{microtype}
\UseMicrotypeSet[protrusion]{basicmath} % disable protrusion for tt fonts
}{}
\usepackage[margin=1in]{geometry}
\usepackage{hyperref}
\hypersetup{unicode=true,
            pdftitle={Data Science Workshop},
            pdfauthor={Alistair Bailey},
            pdfborder={0 0 0},
            breaklinks=true}
\urlstyle{same}  % don't use monospace font for urls
\usepackage{natbib}
\bibliographystyle{apalike}
\usepackage{color}
\usepackage{fancyvrb}
\newcommand{\VerbBar}{|}
\newcommand{\VERB}{\Verb[commandchars=\\\{\}]}
\DefineVerbatimEnvironment{Highlighting}{Verbatim}{commandchars=\\\{\}}
% Add ',fontsize=\small' for more characters per line
\usepackage{framed}
\definecolor{shadecolor}{RGB}{248,248,248}
\newenvironment{Shaded}{\begin{snugshade}}{\end{snugshade}}
\newcommand{\KeywordTok}[1]{\textcolor[rgb]{0.13,0.29,0.53}{\textbf{#1}}}
\newcommand{\DataTypeTok}[1]{\textcolor[rgb]{0.13,0.29,0.53}{#1}}
\newcommand{\DecValTok}[1]{\textcolor[rgb]{0.00,0.00,0.81}{#1}}
\newcommand{\BaseNTok}[1]{\textcolor[rgb]{0.00,0.00,0.81}{#1}}
\newcommand{\FloatTok}[1]{\textcolor[rgb]{0.00,0.00,0.81}{#1}}
\newcommand{\ConstantTok}[1]{\textcolor[rgb]{0.00,0.00,0.00}{#1}}
\newcommand{\CharTok}[1]{\textcolor[rgb]{0.31,0.60,0.02}{#1}}
\newcommand{\SpecialCharTok}[1]{\textcolor[rgb]{0.00,0.00,0.00}{#1}}
\newcommand{\StringTok}[1]{\textcolor[rgb]{0.31,0.60,0.02}{#1}}
\newcommand{\VerbatimStringTok}[1]{\textcolor[rgb]{0.31,0.60,0.02}{#1}}
\newcommand{\SpecialStringTok}[1]{\textcolor[rgb]{0.31,0.60,0.02}{#1}}
\newcommand{\ImportTok}[1]{#1}
\newcommand{\CommentTok}[1]{\textcolor[rgb]{0.56,0.35,0.01}{\textit{#1}}}
\newcommand{\DocumentationTok}[1]{\textcolor[rgb]{0.56,0.35,0.01}{\textbf{\textit{#1}}}}
\newcommand{\AnnotationTok}[1]{\textcolor[rgb]{0.56,0.35,0.01}{\textbf{\textit{#1}}}}
\newcommand{\CommentVarTok}[1]{\textcolor[rgb]{0.56,0.35,0.01}{\textbf{\textit{#1}}}}
\newcommand{\OtherTok}[1]{\textcolor[rgb]{0.56,0.35,0.01}{#1}}
\newcommand{\FunctionTok}[1]{\textcolor[rgb]{0.00,0.00,0.00}{#1}}
\newcommand{\VariableTok}[1]{\textcolor[rgb]{0.00,0.00,0.00}{#1}}
\newcommand{\ControlFlowTok}[1]{\textcolor[rgb]{0.13,0.29,0.53}{\textbf{#1}}}
\newcommand{\OperatorTok}[1]{\textcolor[rgb]{0.81,0.36,0.00}{\textbf{#1}}}
\newcommand{\BuiltInTok}[1]{#1}
\newcommand{\ExtensionTok}[1]{#1}
\newcommand{\PreprocessorTok}[1]{\textcolor[rgb]{0.56,0.35,0.01}{\textit{#1}}}
\newcommand{\AttributeTok}[1]{\textcolor[rgb]{0.77,0.63,0.00}{#1}}
\newcommand{\RegionMarkerTok}[1]{#1}
\newcommand{\InformationTok}[1]{\textcolor[rgb]{0.56,0.35,0.01}{\textbf{\textit{#1}}}}
\newcommand{\WarningTok}[1]{\textcolor[rgb]{0.56,0.35,0.01}{\textbf{\textit{#1}}}}
\newcommand{\AlertTok}[1]{\textcolor[rgb]{0.94,0.16,0.16}{#1}}
\newcommand{\ErrorTok}[1]{\textcolor[rgb]{0.64,0.00,0.00}{\textbf{#1}}}
\newcommand{\NormalTok}[1]{#1}
\usepackage{longtable,booktabs}
\usepackage{graphicx,grffile}
\makeatletter
\def\maxwidth{\ifdim\Gin@nat@width>\linewidth\linewidth\else\Gin@nat@width\fi}
\def\maxheight{\ifdim\Gin@nat@height>\textheight\textheight\else\Gin@nat@height\fi}
\makeatother
% Scale images if necessary, so that they will not overflow the page
% margins by default, and it is still possible to overwrite the defaults
% using explicit options in \includegraphics[width, height, ...]{}
\setkeys{Gin}{width=\maxwidth,height=\maxheight,keepaspectratio}
\IfFileExists{parskip.sty}{%
\usepackage{parskip}
}{% else
\setlength{\parindent}{0pt}
\setlength{\parskip}{6pt plus 2pt minus 1pt}
}
\setlength{\emergencystretch}{3em}  % prevent overfull lines
\providecommand{\tightlist}{%
  \setlength{\itemsep}{0pt}\setlength{\parskip}{0pt}}
\setcounter{secnumdepth}{5}
% Redefines (sub)paragraphs to behave more like sections
\ifx\paragraph\undefined\else
\let\oldparagraph\paragraph
\renewcommand{\paragraph}[1]{\oldparagraph{#1}\mbox{}}
\fi
\ifx\subparagraph\undefined\else
\let\oldsubparagraph\subparagraph
\renewcommand{\subparagraph}[1]{\oldsubparagraph{#1}\mbox{}}
\fi

%%% Use protect on footnotes to avoid problems with footnotes in titles
\let\rmarkdownfootnote\footnote%
\def\footnote{\protect\rmarkdownfootnote}

%%% Change title format to be more compact
\usepackage{titling}

% Create subtitle command for use in maketitle
\newcommand{\subtitle}[1]{
  \posttitle{
    \begin{center}\large#1\end{center}
    }
}

\setlength{\droptitle}{-2em}
  \title{Data Science Workshop}
  \pretitle{\vspace{\droptitle}\centering\huge}
  \posttitle{\par}
\subtitle{British Society for Proteomic Research Meeting 2018}
  \author{Alistair Bailey}
  \preauthor{\centering\large\emph}
  \postauthor{\par}
  \predate{\centering\large\emph}
  \postdate{\par}
  \date{June 08 2018}

\usepackage{booktabs}

% Preamble
\usepackage[none]{hyphenat}
\usepackage[default,osfigures,scale=0.95]{opensans} % Open sans font
\usepackage[T1]{fontenc} % Use 8-bit encoding that has 256 glyphs
\usepackage{lettrine} % The lettrine is the first enlarged letter at the beginning of the text
\raggedbottom 
\usepackage{makeidx} % These lines add bibliography to TOC
\makeindex
\usepackage[nottoc]{tocbibind}
\renewcommand{\bibname}{References} % Rename biblography as References

\usepackage{amsthm}
\newtheorem{theorem}{Theorem}[chapter]
\newtheorem{lemma}{Lemma}[chapter]
\theoremstyle{definition}
\newtheorem{definition}{Definition}[chapter]
\newtheorem{corollary}{Corollary}[chapter]
\newtheorem{proposition}{Proposition}[chapter]
\theoremstyle{definition}
\newtheorem{example}{Example}[chapter]
\theoremstyle{definition}
\newtheorem{exercise}{Exercise}[chapter]
\theoremstyle{remark}
\newtheorem*{remark}{Remark}
\newtheorem*{solution}{Solution}
\begin{document}
\maketitle

{
\setcounter{tocdepth}{1}
\tableofcontents
}
\chapter*{Overview}\label{overview}
\addcontentsline{toc}{chapter}{Overview}

These lessons cover:

\begin{enumerate}
\def\labelenumi{\arabic{enumi}.}
\tightlist
\item
  An introduction to R and RStudio
\item
  An introduction to the tidyverse
\item
  Importing and transforming proteomics data
\item
  Visualisation of proteomics analysis
\end{enumerate}

The analysis is of an example data set of observations for 7702 proteins
from cells in three control experiments and three treatment experiments.
The observations are signal intensity measurements from the mass
spectrometer. These intensities relate to the amount of protein in each
experiment and under each condition. The analysis transforms the data to
examine the effect of treatment on the cellular proteome and visualise
the output using a volcano plot and a heatmap. Click here to download
the csv file.

\section*{Requirements}\label{requirements}
\addcontentsline{toc}{section}{Requirements}

Up to date version of R \citep{R-base} and Rstudio
\citep{rstudioteam2018}

The following R packages:

\begin{Shaded}
\begin{Highlighting}[]
\KeywordTok{install.packages}\NormalTok{(}\KeywordTok{c}\NormalTok{(}\StringTok{"tidyverse"}\NormalTok{,}\StringTok{"gplots"}\NormalTok{,}\StringTok{"pheatmap"}\NormalTok{))}
\end{Highlighting}
\end{Shaded}

\chapter{Introduction}\label{intro}

Placeholder

\section{What are R and RStudio?}\label{what-are-r-and-rstudio}

\subsection{Environments}\label{environments}

\section{Why learn R, or any language
?}\label{why-learn-r-or-any-language}

\section{Finding your way around
RStudio}\label{finding-your-way-around-rstudio}

\subsection{What is real?}\label{what-is-real}

\section{Where am I?}\label{where-am-i}

\section{R projects}\label{r-projects}

\section{Naming things}\label{names}

\section{Seeking help}\label{seeking-help}

\subsection{Asking for help}\label{asking-for-help}

\chapter{Getting started in R and the tidyverse}\label{tidyverse}

Placeholder

\section{Tidy data and the tidyverse}\label{tidy-data-and-the-tidyverse}

\section{Data visualisation}\label{data-visualisation}

\section{Workflow basics}\label{workflow-basics}

\subsection{Assigning objects}\label{assigning-objects}

\subsection{Calling functions}\label{calling-functions}

\subsection{Atomic vectors}\label{atomic-vectors}

\subsection{Attributes}\label{attributes}

\subsection{Factors}\label{factors}

\subsection{Lists}\label{lists}

\subsection{Matrices and arrays}\label{matrices-and-arrays}

\subsection{Data frames}\label{data-frames}

\section{Learning more R}\label{learning-more-r}

\chapter{Creating scripts and importing data}\label{import}

Placeholder

\section{Some definitions}\label{some-definitions}

\section{Using scripts}\label{using-scripts}

\section{Running code}\label{running-code}

\section{Creating a R script}\label{creating-a-r-script}

\section{Setting up our environment}\label{setting-up-our-environment}

\subsection{Bioconductor}\label{bioconductor}

\section{Importing data}\label{importing-data}

\section{Exploring the data}\label{exploring-the-data}

\chapter{Transformation and visualisation}\label{transform}

Placeholder

\section{Fold change and log-fold
change}\label{fold-change-and-log-fold-change}

\section{Dealing with missing values and
normalisation}\label{normalisation}

\section{Heatmap transformation}\label{heatmap-transformation}

\section{Visualising proteomics data}\label{visualising-proteomics-data}

\section{Creating a volcano plot}\label{creating-a-volcano-plot}

\section{Creating a heatmap}\label{creating-a-heatmap}

\chapter{Going further}\label{going-further}

\section{Getting help and joining the R
community}\label{getting-help-and-joining-the-r-community}

\section{Communication: creating reports, presentations and
websites}\label{communication-creating-reports-presentations-and-websites}

\bibliography{packages.bib,book.bib}


\end{document}
