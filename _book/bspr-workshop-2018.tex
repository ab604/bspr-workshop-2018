\documentclass[12pt,]{book}
\usepackage{lmodern}
\usepackage{setspace}
\setstretch{1.5}
\usepackage{amssymb,amsmath}
\usepackage{ifxetex,ifluatex}
\usepackage{fixltx2e} % provides \textsubscript
\ifnum 0\ifxetex 1\fi\ifluatex 1\fi=0 % if pdftex
  \usepackage[T1]{fontenc}
  \usepackage[utf8]{inputenc}
\else % if luatex or xelatex
  \ifxetex
    \usepackage{mathspec}
  \else
    \usepackage{fontspec}
  \fi
  \defaultfontfeatures{Ligatures=TeX,Scale=MatchLowercase}
\fi
% use upquote if available, for straight quotes in verbatim environments
\IfFileExists{upquote.sty}{\usepackage{upquote}}{}
% use microtype if available
\IfFileExists{microtype.sty}{%
\usepackage{microtype}
\UseMicrotypeSet[protrusion]{basicmath} % disable protrusion for tt fonts
}{}
\usepackage[margin=1in]{geometry}
\usepackage{hyperref}
\hypersetup{unicode=true,
            pdftitle={Data Science Workshop},
            pdfauthor={Alistair Bailey},
            pdfborder={0 0 0},
            breaklinks=true}
\urlstyle{same}  % don't use monospace font for urls
\usepackage{natbib}
\bibliographystyle{apalike}
\usepackage{color}
\usepackage{fancyvrb}
\newcommand{\VerbBar}{|}
\newcommand{\VERB}{\Verb[commandchars=\\\{\}]}
\DefineVerbatimEnvironment{Highlighting}{Verbatim}{commandchars=\\\{\}}
% Add ',fontsize=\small' for more characters per line
\usepackage{framed}
\definecolor{shadecolor}{RGB}{248,248,248}
\newenvironment{Shaded}{\begin{snugshade}}{\end{snugshade}}
\newcommand{\KeywordTok}[1]{\textcolor[rgb]{0.13,0.29,0.53}{\textbf{#1}}}
\newcommand{\DataTypeTok}[1]{\textcolor[rgb]{0.13,0.29,0.53}{#1}}
\newcommand{\DecValTok}[1]{\textcolor[rgb]{0.00,0.00,0.81}{#1}}
\newcommand{\BaseNTok}[1]{\textcolor[rgb]{0.00,0.00,0.81}{#1}}
\newcommand{\FloatTok}[1]{\textcolor[rgb]{0.00,0.00,0.81}{#1}}
\newcommand{\ConstantTok}[1]{\textcolor[rgb]{0.00,0.00,0.00}{#1}}
\newcommand{\CharTok}[1]{\textcolor[rgb]{0.31,0.60,0.02}{#1}}
\newcommand{\SpecialCharTok}[1]{\textcolor[rgb]{0.00,0.00,0.00}{#1}}
\newcommand{\StringTok}[1]{\textcolor[rgb]{0.31,0.60,0.02}{#1}}
\newcommand{\VerbatimStringTok}[1]{\textcolor[rgb]{0.31,0.60,0.02}{#1}}
\newcommand{\SpecialStringTok}[1]{\textcolor[rgb]{0.31,0.60,0.02}{#1}}
\newcommand{\ImportTok}[1]{#1}
\newcommand{\CommentTok}[1]{\textcolor[rgb]{0.56,0.35,0.01}{\textit{#1}}}
\newcommand{\DocumentationTok}[1]{\textcolor[rgb]{0.56,0.35,0.01}{\textbf{\textit{#1}}}}
\newcommand{\AnnotationTok}[1]{\textcolor[rgb]{0.56,0.35,0.01}{\textbf{\textit{#1}}}}
\newcommand{\CommentVarTok}[1]{\textcolor[rgb]{0.56,0.35,0.01}{\textbf{\textit{#1}}}}
\newcommand{\OtherTok}[1]{\textcolor[rgb]{0.56,0.35,0.01}{#1}}
\newcommand{\FunctionTok}[1]{\textcolor[rgb]{0.00,0.00,0.00}{#1}}
\newcommand{\VariableTok}[1]{\textcolor[rgb]{0.00,0.00,0.00}{#1}}
\newcommand{\ControlFlowTok}[1]{\textcolor[rgb]{0.13,0.29,0.53}{\textbf{#1}}}
\newcommand{\OperatorTok}[1]{\textcolor[rgb]{0.81,0.36,0.00}{\textbf{#1}}}
\newcommand{\BuiltInTok}[1]{#1}
\newcommand{\ExtensionTok}[1]{#1}
\newcommand{\PreprocessorTok}[1]{\textcolor[rgb]{0.56,0.35,0.01}{\textit{#1}}}
\newcommand{\AttributeTok}[1]{\textcolor[rgb]{0.77,0.63,0.00}{#1}}
\newcommand{\RegionMarkerTok}[1]{#1}
\newcommand{\InformationTok}[1]{\textcolor[rgb]{0.56,0.35,0.01}{\textbf{\textit{#1}}}}
\newcommand{\WarningTok}[1]{\textcolor[rgb]{0.56,0.35,0.01}{\textbf{\textit{#1}}}}
\newcommand{\AlertTok}[1]{\textcolor[rgb]{0.94,0.16,0.16}{#1}}
\newcommand{\ErrorTok}[1]{\textcolor[rgb]{0.64,0.00,0.00}{\textbf{#1}}}
\newcommand{\NormalTok}[1]{#1}
\usepackage{longtable,booktabs}
\usepackage{graphicx,grffile}
\makeatletter
\def\maxwidth{\ifdim\Gin@nat@width>\linewidth\linewidth\else\Gin@nat@width\fi}
\def\maxheight{\ifdim\Gin@nat@height>\textheight\textheight\else\Gin@nat@height\fi}
\makeatother
% Scale images if necessary, so that they will not overflow the page
% margins by default, and it is still possible to overwrite the defaults
% using explicit options in \includegraphics[width, height, ...]{}
\setkeys{Gin}{width=\maxwidth,height=\maxheight,keepaspectratio}
\IfFileExists{parskip.sty}{%
\usepackage{parskip}
}{% else
\setlength{\parindent}{0pt}
\setlength{\parskip}{6pt plus 2pt minus 1pt}
}
\setlength{\emergencystretch}{3em}  % prevent overfull lines
\providecommand{\tightlist}{%
  \setlength{\itemsep}{0pt}\setlength{\parskip}{0pt}}
\setcounter{secnumdepth}{5}
% Redefines (sub)paragraphs to behave more like sections
\ifx\paragraph\undefined\else
\let\oldparagraph\paragraph
\renewcommand{\paragraph}[1]{\oldparagraph{#1}\mbox{}}
\fi
\ifx\subparagraph\undefined\else
\let\oldsubparagraph\subparagraph
\renewcommand{\subparagraph}[1]{\oldsubparagraph{#1}\mbox{}}
\fi

%%% Use protect on footnotes to avoid problems with footnotes in titles
\let\rmarkdownfootnote\footnote%
\def\footnote{\protect\rmarkdownfootnote}

%%% Change title format to be more compact
\usepackage{titling}

% Create subtitle command for use in maketitle
\newcommand{\subtitle}[1]{
  \posttitle{
    \begin{center}\large#1\end{center}
    }
}

\setlength{\droptitle}{-2em}

  \title{Data Science Workshop}
    \pretitle{\vspace{\droptitle}\centering\huge}
  \posttitle{\par}
  \subtitle{British Society for Proteomic Research Meeting 2018}
  \author{Alistair Bailey}
    \preauthor{\centering\large\emph}
  \postauthor{\par}
      \predate{\centering\large\emph}
  \postdate{\par}
    \date{July 12 2018}

\usepackage{booktabs}

% Preamble
\usepackage[none]{hyphenat}
\usepackage[default,osfigures,scale=0.95]{opensans} % Open sans font
\usepackage[T1]{fontenc} % Use 8-bit encoding that has 256 glyphs
\usepackage{lettrine} % The lettrine is the first enlarged letter at the beginning of the text
\raggedbottom 
\usepackage{makeidx} % These lines add bibliography to TOC
\makeindex
\usepackage[nottoc]{tocbibind}
\renewcommand{\bibname}{References} % Rename biblography as References

\usepackage{amsthm}
\newtheorem{theorem}{Theorem}[chapter]
\newtheorem{lemma}{Lemma}[chapter]
\theoremstyle{definition}
\newtheorem{definition}{Definition}[chapter]
\newtheorem{corollary}{Corollary}[chapter]
\newtheorem{proposition}{Proposition}[chapter]
\theoremstyle{definition}
\newtheorem{example}{Example}[chapter]
\theoremstyle{definition}
\newtheorem{exercise}{Exercise}[chapter]
\theoremstyle{remark}
\newtheorem*{remark}{Remark}
\newtheorem*{solution}{Solution}
\begin{document}
\maketitle

{
\setcounter{tocdepth}{1}
\tableofcontents
}
\chapter*{Overview}\label{overview}
\addcontentsline{toc}{chapter}{Overview}

This book covers:

\begin{enumerate}
\def\labelenumi{\arabic{enumi}.}
\tightlist
\item
  An introduction to R and RStudio
\item
  An introduction to tidyverse and base R
\item
  Importing and transforming proteomics data
\item
  Visualisation of proteomics analysis
\end{enumerate}

The analysis is of an example data set of observations for 7702 proteins
from cells in three control experiments and three treatment experiments.
The observations are signal intensity measurements from the mass
spectrometer. These intensities relate the concentration of protein
observed in each experiment and under each condition. The analysis
transforms the data to examine the effect of treatment on the cellular
proteome and visualise the output using a volcano plot , a heatmap, a
Venn diagram and peptide sequence logos. Click here to download the csv
file.

\section*{Requirements}\label{requirements}
\addcontentsline{toc}{section}{Requirements}

An up to date version of R \citep{R-base} and RStudio
\citep{rstudioteam2018}.

If you are new to R, then the first thing to know is that R is a
programming language and RStudio is a program for working with R called
an integrated development environment (IDE). You can use R without
RStudio, but not the other way around. Further details in Chapter
\ref{r-rstudio}.

\href{https://cran.r-project.org/}{Download R here} and
\href{https://www.rstudio.com/products/rstudio/download/}{Download
RStudio Desktop here}.

These materials were generated using R version 3.5.0.

Once you've installed R and RStudio, you'll also need a few R packages.
Packages are collections of
\protect\hyperlink{function-anatomy}{functions}.

Open RStudio and put the code below into the \texttt{Console} window and
press \texttt{Enter} to install these three packages.

\begin{Shaded}
\begin{Highlighting}[]
\KeywordTok{install.packages}\NormalTok{(}\KeywordTok{c}\NormalTok{(}\StringTok{"plyr"}\NormalTok{,}\StringTok{"tidyverse"}\NormalTok{,}\StringTok{"gplots"}\NormalTok{,}\StringTok{"pheatmap"}\NormalTok{,}
                   \StringTok{"gridExtra"}\NormalTok{,}\StringTok{"VennDiagram"}\NormalTok{,}\StringTok{"ggseqlogo"}\NormalTok{))}
\end{Highlighting}
\end{Shaded}

\chapter{Introduction}\label{intro}

There are many resources for learning R on the web. Much of Chapters
\ref{intro}, \ref{tidyverse}, \ref{import} and \ref{dplyr} derive from a
\href{http://www.datacarpentry.org/lessons/}{Data Carpentry lesson}
using ecological data that I have previously
\href{https://southampton-rsg.github.io/2017-08-01-southampton-dc/novice/R-ecology-lesson/index.html}{reworked},
which in turn takes a lot from \href{http://r4ds.had.co.nz/}{Hadley
Wickham's R for Data Science} aka \textbf{R4DS}. Follow the links to
access those materials.

Chapter \ref{transform} deals with some statistical transformations and
visualisation methods in the context of proteomics data.

Whilst finally in Chapter \ref{going-further} there is some advice about
how to build upon the materials covered here.

In terms of philosophy:

\begin{enumerate}
\def\labelenumi{\arabic{enumi}.}
\item
  The primary motivation for using tools such as R is to get more done,
  in less time and with less pain.
\item
  And the overall aim is to \emph{understand and communicate} findings
  from our data.
\end{enumerate}



\begin{figure}

{\centering \includegraphics[width=0.8\linewidth]{img/data_project_pipeline} 

}

\caption{Data project workflow.}\label{fig:pipeline}
\end{figure}

As shown in Figure \ref{fig:pipeline} of typical data analysis workflow,
to acheive this aim we need to learn tools that enable us to perform the
fundamental tasks of tasks of importing, tidying and often transforming
the data. Transformation means for example, selecting a subset of the
data to work with, or calculating the mean of a set of observations.
We'll cover that in Chapter \ref{transform}.

But first\ldots{}

\section{What are R and RStudio?}\label{r-rstudio}

\textbf{\emph{``There are only two kinds of languages: the ones people
complain about and the ones nobody uses''}}

\emph{Bjarne Stroustrup}

\textbf{R} is a programming language that follows the philosophy laid
down by it's predecessor S. The philosophy being that users begin in an
interactive environment where they don't consciously think of themselves
as programming. It was created in 1993, and documented in
\citep{ihaka1996}.

Reasons R has become popular include that it is both open source and
cross platform, and that it has broad functionality, from the analysis
of data and creating powerful graphical visualisations and web apps.

Like all languages though it has limitations, for example the syntax is
initially confusing.

Take for example the word \texttt{environment}\ldots{}

\subsection{Environments}\label{environments}

An environment is where we bring our data to work with it. Here we work
in a R envrionment, using the R language as a set of tools.
\textbf{RStudio} is an integrated development environment, or IDE for R
programming. It is regularly updated, and upgrading enables access to
the latest features.

The latest version can be downloaded here:
\url{http://www.rstudio.com/download}

\section{Why learn R, or any language
?}\label{why-learn-r-or-any-language}

We can write R code without saving it, but it's generally more useful to
write and save our code as a script. Working with scripts makes the
steps you used in your analysis clear, and the code you write can be
inspected by someone else who can give you feedback and spot mistakes.

Learning R (or any programming language) and working with scripts forces
you to have deeper understanding of what you are doing, facilitates your
learning and comprehension of the methods you use:

\begin{itemize}
\tightlist
\item
  Writing and publishing code is important for reproducible resarch
\item
  R has many thousands of packages covering many disciplines.
\item
  R can work with many types of data.
\item
  They is a large R community for development and support.
\item
  Using R gives you control over your figures and reports.
\end{itemize}

\section{Finding your way around
RStudio}\label{finding-your-way-around-rstudio}

Let's begin by learning about \href{https://www.rstudio.com/}{RStudio},
the Integrated Development Environment (IDE).

We will use R Studio IDE to write code, navigate the files found on our
computer, inspect the variables we are going to create, and visualize
the plots we will generate. R Studio can also be used for other things
(e.g., version control, developing packages, writing Shiny apps) that we
don't have time to cover during this workshop.

R Studio is divided into ``Panes'', see Figure \ref{fig:rstudio}.

When you first open it, there are three panes,the console where you type
commands, your environment/history (top-right), and your
files/plots/packages/help/viewer (bottom-right).

The enivronment shows all the R objects you have created or are using,
such as data you have imported.

The output pane can be used to view any plots you have created.

Not opened at first start up is the fourth default pane: the script
editor pane, but this will open as soon as we create/edit a R script (or
many other document types). \emph{The script editor is where will be
typing much of the time.}



\begin{figure}

{\centering \includegraphics[width=0.8\linewidth]{img/rstudio_ide_image} 

}

\caption{The Rstudio Integrated Development Environment (IDE).}\label{fig:rstudio}
\end{figure}

The placement of these panes and their content can be customized (see
menu, R Studio -\textgreater{} Tools -\textgreater{} Global Options
-\textgreater{} Pane Layout). One of the advantages of using R Studio is
that all the information you need to write code is available ina single
window. Additionally, with many shortcuts, auto-completion, and
highlighting for the major file types you use while developing in R, R
Studio will make typing easier and less error-prone.

Time for a philosphical diversion\ldots{}

\subsection{What is real?}\label{what-is-real}

At the start, we might consider our environment ``real'' - that is to
say the objects we've created/loaded and are using are ``real''. But
it's much better in the long run to consider our scripts as ``real'' -
our scripts are where we write down the code that creates our objects
that we'll be using in our environment.

\textbf{As a script is a document, it is reproducible}

Or to put it another way: we can easily recreate an environment from our
scripts, but not so easily create a script from an enivronment.

To support this notion of thinking in terms of our scripts as real, we
recommend turning off the preservation of workspaces between sessions by
setting the Tools \textgreater{} Global Options menu in R studio as
shown in Figure \ref{fig:workspace}:



\begin{figure}

{\centering \includegraphics[width=0.8\linewidth]{img/rdata_turn_off} 

}

\caption{Don't save your workspace, save your script instead.}\label{fig:workspace}
\end{figure}

\section{Where am I?}\label{where-am-i}

R studio tells you where you are in terms of directory address like so:



\begin{figure}

{\centering \includegraphics[width=0.8\linewidth]{img/rstudio_working_directory} 

}

\caption{Your working directory}\label{fig:working-directory}
\end{figure}

If you are unfamiliar with how computers structure folders and files,
then consider a tree with a root from which the trunk extends and
branches divide. In the image above, the \textasciitilde{} symbol
represents a contraction of the path from the root to the `home'
directory (in Windows this is `Documents') and then the forward slashes
are the branches. (Note: Windows uses backslashes, Unix type systems and
R use forwardslashes).

It is good practice to keep a set of related data, analyses, and text
self-contained in a single folder, called the \textbf{working
directory}. All of the scripts within this folder can then use
\emph{relative paths} to files that indicate where inside the project a
file is located (as opposed to absolute paths, which point to where a
file is on a specific computer). Working this way makes it a lot easier
to move your project around on your computer and share it with others
without worrying about whether or not the underlying scripts will still
work.



\begin{figure}

{\centering \includegraphics[width=0.8\linewidth]{img/R-ecology-work_dir_structure} 

}

\caption{A typical directory structure}\label{fig:dir-structure}
\end{figure}

\section{R projects}\label{r-projects}

RStudio also has a facility to keep all files associated with a
particular analysis together called a project.

Creating a project creates a working directory for you and also
remembers its location (allowing you to quickly navigate to it) and
optionally preserves custom settings and open files to make it easier to
resume work after a break.



\begin{figure}

{\centering \includegraphics[width=0.8\linewidth]{img/rstudio_create_project} 

}

\caption{Creating a R project}\label{fig:r-projects}
\end{figure}

Below, we will go through the steps for creating an ``R Project'':

\begin{itemize}
\tightlist
\item
  Start R Studio (presentation of R Studio -below- should happen here)
\item
  Under the \texttt{File} menu, click on \texttt{New\ project}, choose
  \texttt{New\ directory}, then \texttt{Empty\ project}
\item
  Enter a name for this new folder (or ``directory'', in computer
  science), and choose a convenient location for it. This will be your
  \textbf{working directory} for the rest of the day (e.g.,
  \texttt{\textasciitilde{}/bspr-workshop})
\item
  Click on ``Create project''
\item
  Under the \texttt{Files} tab on the right of the screen, click on
  \texttt{New\ Folder} and create a folder named \texttt{data} within
  your newly created working directory. (e.g.,
  \texttt{\textasciitilde{}/bspr-workshopdata})
\item
  Create a new R script (File \textgreater{} New File \textgreater{} R
  script) and save it in your working directory (e.g.
  \texttt{bspr-workshop-script.R})
\end{itemize}

\hypertarget{names}{\section{Naming things}\label{names}}

\href{https://ropensci.org/blog/2017/12/08/rprofile-jenny-bryan/}{Jenny
Bryan} has three principles for
\href{http://www2.stat.duke.edu/~rcs46/lectures_2015/01-markdown-git/slides/naming-slides/naming-slides.pdf}{naming
things} that are well worth remembering.

When you names something, a file or an object, ideally it should be:

\begin{enumerate}
\def\labelenumi{\arabic{enumi}.}
\tightlist
\item
  Machine readable (no whitespace, punctuation, upper AND
  lowercase\ldots{})
\item
  Human readable (makes sense in 6 months or 2 years time)
\item
  Plays well with default ordering (numerical or date order)
\end{enumerate}

\section{Seeking help}\label{seeking-help}

If you need help with a specific R function, let's say
\texttt{barplot()}, you can type:

\begin{Shaded}
\begin{Highlighting}[]
\NormalTok{?barplot}
\end{Highlighting}
\end{Shaded}

If you can't find what you are looking for, you can use the
\href{http://www.rdocumentation.org}{rdocumention.org} website that
searches through the help files across all packages available.

A Google or internet search ``R \textless{}task\textgreater{}'' will
often either send you to the appropriate package documentation or a
helpful forum question that someone else already asked, such as
\href{http://stackoverflow.com/questions/tagged/r}{Stack Overflow} or
the \href{https://community.rstudio.com/}{RStudio Community}.

\subsection{Asking for help}\label{asking-for-help}

As well as knowing
\href{https://www.tidyverse.org/help/\#where-to-ask}{where to ask}, the
key to get help from someone is for them to grasp your problem rapidly.
You should make it as easy as possible to pinpoint where the issue might
be.

Try to use the correct words to describe your problem. For instance, a
package is not the same thing as a library. Most people will understand
what you meant, but others have really strong feelings about the
difference in meaning. The key point is that it can make things
confusing for people trying to help you. Be as precise as possible when
describing your problem.

If possible, try to reduce what doesn't work to a simple
\emph{reproducible example} otherwise known as a \emph{reprex}.

For more information on how to write a reproducible example see
\href{https://www.tidyverse.org/help/\#reprex}{this article}.

\chapter{Getting started in R and the tidyverse}\label{tidyverse}

Functions are a way to automate common tasks and R comes with a set of
functions called the \texttt{base} package. We will be using some
\texttt{base} functions in Chapter \ref{transform}, but to introduce the
concept of using \protect\hyperlink{function-anatomy}{functions} we'll
begin with the \texttt{tidyverse}.

\section{The tidyverse and tidy data}\label{the-tidyverse-and-tidy-data}

The \href{https://www.tidyverse.org/}{tidyverse} \citep{R-tidyverse} is
\emph{``an opinionated collection of R packages designed for data
science''} .

Tidyverse packages contain functions that \emph{``share an underlying
design philosophy, grammar, and data structures.''} It's this
philiosophy that makes tidyverse functions and packages relatively easy
to learn and use.

Tidy data follows three principals for tabular data as proposed in the
Tidy Data paper \url{http://www.jstatsoft.org/v59/i10/paper} :

\begin{enumerate}
\def\labelenumi{\arabic{enumi}.}
\tightlist
\item
  Every variable has its own column.
\item
  Every observation has its own row.
\item
  Each value has its own cell.
\end{enumerate}

If our table was proteomics data then, we might have a set of variables
such as the peptide sequence, mass or length observed for a number of
peptides. Therefore each peptide would have a row with columns for
peptide sequence, mass and length with the value for each variable in
separate cells, as seen in Figure \ref{fig:tidy-prot}.



\begin{figure}

{\centering \includegraphics[width=0.8\linewidth]{img/tidy_prot_data} 

}

\caption{An example of tidy proteomics data}\label{fig:tidy-prot}
\end{figure}

Often much of the work in any data analysis is getting our data into a
tidy form.

We can't do everything in the tidyverse, and everything we can do in the
tidyverse can be done in what is called base R or other packages, but
the motivation behind the tidyverse is to ease the pain of data
manipulation.

With this in mind, the two tasks we are most likely to want to do in
data science are:

\begin{enumerate}
\def\labelenumi{\arabic{enumi}.}
\tightlist
\item
  Visualise our data
\item
  Automate our processes.
\end{enumerate}

Taking our cue from \href{http://r4ds.had.co.nz/}{R4DS} let's try an
example.

\section{Data visualisation}\label{data-visualisation}

The \texttt{ggplot2} package implements the \emph{grammer of graphics},
for describing and building graphs.

The motivation here is twofold:

\begin{enumerate}
\def\labelenumi{\arabic{enumi}.}
\tightlist
\item
  To begin to grasp the grammar of graphics approach to creating plots.
  This will be our first example of automating a task using a function.
\item
  To demonstrate how plotting is often the most useful thing we can do
  when trying to understand our data.
\end{enumerate}

We'll use the \texttt{mpg} dataset that comes with the tidyverse to
examine the question \emph{do cars with big engines use more fuel than
cars with small engines?}

Try \texttt{?mpg} to learn more about the data.

\begin{enumerate}
\def\labelenumi{\arabic{enumi}.}
\tightlist
\item
  Engine size in litres is in the \texttt{displ} column.
\item
  Fuel efficiency on the highway in miles per gallon is given in the
  \texttt{hwy} column.
\end{enumerate}

To create a plot of engine size \texttt{displ} (x-axis) against fuel
efficiency \texttt{hwy} (y-axis) we do the following:

\begin{enumerate}
\def\labelenumi{\arabic{enumi}.}
\tightlist
\item
  Use the \texttt{ggplot()} function to create an empty graph.
\item
  Provide ggplot with a first input or \textbf{argument} of the data
  (here \texttt{mpg}).
\item
  Then we follow the ggplot function with a \texttt{+} sign to indicate
  we are going to add more code, followed by a \texttt{geom\_point()}
  function to add a layer of points mapping some aesthetics for the x
  and y axes.
\item
  Mapping is always paired to aesthetics \texttt{aes()}. An aesthetic is
  a visual property of the objects in your plot, such a point size,
  shape or point colour.
\end{enumerate}

Therefore to plot engine size (x-axis) against fuel efficiency (y-axis)
we use the following code:

\begin{Shaded}
\begin{Highlighting}[]
\KeywordTok{ggplot}\NormalTok{(}\DataTypeTok{data =}\NormalTok{ mpg) }\OperatorTok{+}\StringTok{ }
\StringTok{  }\KeywordTok{geom_point}\NormalTok{(}\DataTypeTok{mapping =} \KeywordTok{aes}\NormalTok{(}\DataTypeTok{x =}\NormalTok{ displ, }\DataTypeTok{y =}\NormalTok{ hwy))}
\end{Highlighting}
\end{Shaded}

\includegraphics{bspr-workshop-2018_files/figure-latex/mpg-plot-1,mpg_point_plot-1.pdf}

This plot shows a negative relationship between engine size and fuel
efficiency.

Now try extending this code to include to add a \texttt{colour}
aesthetic to the the \texttt{aes()} function, let
\texttt{colour\ =\ class}, \texttt{class} being the veichle type. This
should create a plot with as before but with the points coloured
according to the viechle type to expand our understanding.

\begin{Shaded}
\begin{Highlighting}[]
\KeywordTok{ggplot}\NormalTok{(}\DataTypeTok{data =}\NormalTok{ mpg) }\OperatorTok{+}\StringTok{ }
\StringTok{  }\KeywordTok{geom_point}\NormalTok{(}\DataTypeTok{mapping =} \KeywordTok{aes}\NormalTok{(}\DataTypeTok{x =}\NormalTok{ displ, }\DataTypeTok{y =}\NormalTok{ hwy, }\DataTypeTok{colour =}\NormalTok{ class))}
\end{Highlighting}
\end{Shaded}

\includegraphics{bspr-workshop-2018_files/figure-latex/mpg-plot-2-1.pdf}

Now we can see that as we might expect, bigger cars such as SUVs tend to
have bigger engines and are also less fuel efficient, but some smaller
cars such as 2-seaters also have big engines and greater fuel
efficiency. Hence we have a more nuanced view with this additional
aesthetic.

Check out the ggplot2 documentation for all the aesthetic possibilities
(and Google for examples): \url{http://ggplot2.tidyverse.org/reference/}

So now we have re-usable code snippet for generating plots in R:

\begin{Shaded}
\begin{Highlighting}[]
\KeywordTok{ggplot}\NormalTok{(}\DataTypeTok{data =} \OperatorTok{<}\NormalTok{DATA}\OperatorTok{>}\NormalTok{) }\OperatorTok{+}\StringTok{ }
\StringTok{  }\ErrorTok{<}\NormalTok{GEOM_FUNCTION}\OperatorTok{>}\NormalTok{(}\DataTypeTok{mapping =} \KeywordTok{aes}\NormalTok{(}\OperatorTok{<}\NormalTok{MAPPINGS}\OperatorTok{>}\NormalTok{))}
\end{Highlighting}
\end{Shaded}

Concretely, in our first example \texttt{\textless{}DATA\textgreater{}}
was \texttt{mpg}, the \texttt{\textless{}GEOM\_FUNCTION\textgreater{}}
was \texttt{geom\_point()} and the arguments we supplies to map our
aesthetics \texttt{\textless{}MAPPINGS\textgreater{}} were
\texttt{x\ =\ displ,\ y\ =\ hwy}.

As we can use this code for any tidy data set, hopefully you are
beginning to see how a small amount of code can do a lot.

\subsection{Visualisations}\label{visualisations}

Claus Wilke has written a very nice guide to visualising data using R
called \href{http://serialmentor.com/dataviz/index.html}{Fundamentals of
Data Visualization}.

\section{Workflow basics}\label{workflow-basics}

Let's run through the basics of working in R to conclude this chapter.

\subsection{Assigning objects}\label{assigning-objects}

Objects are just a way to store data inside the R environment. We create
objects using the assignment operator \texttt{\textless{}-}:

\begin{Shaded}
\begin{Highlighting}[]
\NormalTok{mass_kg <-}\StringTok{ }\DecValTok{55}
\end{Highlighting}
\end{Shaded}

Read this as \emph{``mass\_kg gets value 55''} in your head.

Using \texttt{\textless{}-} can be annoying to type, so use RStudio's
keyboard short cut: Alt + - (the minus sign) to make life easier.

Many people ask why we use this assignment operator when we can use
\texttt{=} instead?

\href{https://twitter.com/_colinfay/status/1006139974377443328}{Colin
Fay had a Twitter thread on this subject}, but the reason I favour most
is that it provides clarity. The arrow points in the direction of the
assigment (it is actually possible to assign in the other direction too)
and it distinguishes between creating an object in the workspace and
assigning a value inside a function.

Object name style is a matter of choice, but must start with a letter
and can only contain letters, numbers, \texttt{\_} and \texttt{.}. We
recommend using descriptive names and using \texttt{\_} between words.
Some special symbols cannot be used in variable names, so watch out for
those.

So here we've used the name to indicate its value represents a mass in
kilograms. Look in your environment pane and you'll see the
\texttt{mass\_kg} object containing the (data) value 55.

We can inspect an object by typing it's name:

\begin{Shaded}
\begin{Highlighting}[]
\NormalTok{mass_kg}
\end{Highlighting}
\end{Shaded}

\begin{verbatim}
## [1] 55
\end{verbatim}

What's wrong here?

\begin{Shaded}
\begin{Highlighting}[]
\NormalTok{mass_KG}
\end{Highlighting}
\end{Shaded}

\texttt{Error:\ object\ \textquotesingle{}mass\_KG\textquotesingle{}\ not\ found}

This error illustrates that typos matter, everything must be precise and
\texttt{mass\_KG} is not the same as \texttt{mass\_kg}.
\texttt{mass\_KG} doesn't exist, hence the error.

\hypertarget{function-anatomy}{\subsection{Function
anatomy}\label{function-anatomy}}

Functions in R are objects followed by parentheses, such as
\texttt{library()}.

Functions have the form:

\texttt{function\_name(arg1\ =\ val,\ arg2\ =\ val2,\ ...)}

The use of arguements or inputs allows us to generalise. That is to say
not just do something in a specific case, but in many cases. For example
not just make a scatter plot for the \texttt{mpg} dataset, but for any
dataset of observations that can be plotted pairwise.

Let's use \texttt{seq()} to create a \textbf{seq}uence of numbers, and
at the same time practice tab completion.

Start typing \texttt{se} in the console and you should see a list of
functions appear, add \texttt{q} to shorten the list, then use the up
and down arrow to highlight the function of interest \texttt{seq()} and
hit Tab to select.

RStudio puts the cursor between the parentheses to prompt us to enter
some arguments. Here we'll use 1 as the start and 10 as the end:

\begin{Shaded}
\begin{Highlighting}[]
\KeywordTok{seq}\NormalTok{(}\DecValTok{1}\NormalTok{,}\DecValTok{10}\NormalTok{)}
\end{Highlighting}
\end{Shaded}

\begin{verbatim}
##  [1]  1  2  3  4  5  6  7  8  9 10
\end{verbatim}

If we left off a parentheses to close the function, then when we hit
enter we'll see a \texttt{+} indicating RStudio is expecting further
code. We either add the missing part or press Escape to cancel the code.

Let's call a function and make an assignment at the same time. Here
we'll use the base R function \texttt{seq()} which takes three
arguments: \texttt{from}, \texttt{to} and \texttt{by}.

Read the following code as *``make an object called my\_sequence that
stores a sequence of numbers from 2 to 20 by intervals of 2*.

\begin{Shaded}
\begin{Highlighting}[]
\NormalTok{my_sequence <-}\StringTok{ }\KeywordTok{seq}\NormalTok{(}\DecValTok{2}\NormalTok{,}\DecValTok{20}\NormalTok{,}\DecValTok{2}\NormalTok{)}
\end{Highlighting}
\end{Shaded}

This time nothing was returned to the console, but we now have an object
called \texttt{my\_sequence} in our environment.

Can you remember how to inspect it?

If we want to subset elements of \texttt{my\_sequence} we use square
brackets \texttt{{[}{]}}.

For example element five would be subset by:

\begin{Shaded}
\begin{Highlighting}[]
\NormalTok{my_sequence[}\DecValTok{5}\NormalTok{]}
\end{Highlighting}
\end{Shaded}

\begin{verbatim}
## [1] 10
\end{verbatim}

Here the number five is the index of the vector, not the value of the
fifth element. The value of the fifth element is 10.

And returning multiple elements uses a colon \texttt{:}, like so

\begin{Shaded}
\begin{Highlighting}[]
\NormalTok{my_sequence[}\DecValTok{5}\OperatorTok{:}\DecValTok{8}\NormalTok{]}
\end{Highlighting}
\end{Shaded}

\begin{verbatim}
## [1] 10 12 14 16
\end{verbatim}

\hypertarget{atomics}{\subsection{Atomic vectors}\label{atomics}}

We actually made an atomic vector already when we made
\texttt{my\_sequence}. We made a a one dimensional group of numbers, in
a sequence from two to twenty.

We're not going to be working much with atomic vectors in this workshop,
but to make you aware of how R stores data, atomic vector types are:

\begin{itemize}
\tightlist
\item
  Doubles: regular numbers, +ve or -ve and with or without decimal
  places. AKA numerics.
\item
  Integers: whole numbers, specified with an upper-case L, e.g.
  \texttt{int\ \textless{}-\ 2L}
\item
  Characters: Strings of text
\item
  Logicals: these store \texttt{TRUE}s and \texttt{FALSE}s which are
  useful for comparisons.
\item
  Complex: this would be a vector of numbers with imaginary terms.
\item
  Raw: these vectors store raw bytes of data.
\end{itemize}

Let's make a character vector and check the type:

\begin{Shaded}
\begin{Highlighting}[]
\NormalTok{cards <-}\StringTok{ }\KeywordTok{c}\NormalTok{(}\StringTok{"ace"}\NormalTok{, }\StringTok{"king"}\NormalTok{, }\StringTok{"queen"}\NormalTok{, }\StringTok{"jack"}\NormalTok{, }\StringTok{"ten"}\NormalTok{)}

\NormalTok{cards}
\end{Highlighting}
\end{Shaded}

\begin{verbatim}
## [1] "ace"   "king"  "queen" "jack"  "ten"
\end{verbatim}

\begin{Shaded}
\begin{Highlighting}[]
\KeywordTok{typeof}\NormalTok{(cards)}
\end{Highlighting}
\end{Shaded}

\begin{verbatim}
## [1] "character"
\end{verbatim}

\subsection{Attributes}\label{attributes}

An attribute is a piece of information you can attach to an object, such
as names or dimensions. Attributes such as dimensions are added when we
create an object, but others such as names can be added.

Let's look at the \texttt{mpg} data frame dimensions:

\begin{Shaded}
\begin{Highlighting}[]
\CommentTok{# mpg has 234 rows (observations) and 11 columns (variables)}
\KeywordTok{dim}\NormalTok{(mpg)}
\end{Highlighting}
\end{Shaded}

\begin{verbatim}
## [1] 234  11
\end{verbatim}

\subsection{Factors}\label{factors}

Factors are Rs way of storing categorical information such as eye colour
or car type. A factor is something that can only have certain values,
and can be ordered (such as \texttt{low},\texttt{medium},\texttt{high})
or unordered such as types of fruit.

Factors are useful as they code string variables such as ``red'' or
``blue'' to integer values e.g.~1 and 2, which can be used in
statistical models and when plotting, but they are confusing as they
look like strings.

\textbf{Factors look like strings, but behave like integers.}

Historically R converts strings to factors when we load and create data,
but it's often not what we want as a default. Fortunately, in the
tidyverse strings are not treated as factors by default.

\subsection{Lists}\label{lists}

Lists also group data into one dimensional sets of data. The difference
being that list group objects instead of individual values, such as
several atomic vectors.

For example, let's make a list containing a vector of numbers and a
character vector

\begin{Shaded}
\begin{Highlighting}[]
\NormalTok{list_}\DecValTok{1}\NormalTok{ <-}\StringTok{ }\KeywordTok{list}\NormalTok{(}\DecValTok{1}\OperatorTok{:}\DecValTok{110}\NormalTok{,}\StringTok{"R"}\NormalTok{)}

\NormalTok{list_}\DecValTok{1}
\end{Highlighting}
\end{Shaded}

\begin{verbatim}
## [[1]]
##   [1]   1   2   3   4   5   6   7   8   9  10  11  12  13  14  15  16  17
##  [18]  18  19  20  21  22  23  24  25  26  27  28  29  30  31  32  33  34
##  [35]  35  36  37  38  39  40  41  42  43  44  45  46  47  48  49  50  51
##  [52]  52  53  54  55  56  57  58  59  60  61  62  63  64  65  66  67  68
##  [69]  69  70  71  72  73  74  75  76  77  78  79  80  81  82  83  84  85
##  [86]  86  87  88  89  90  91  92  93  94  95  96  97  98  99 100 101 102
## [103] 103 104 105 106 107 108 109 110
## 
## [[2]]
## [1] "R"
\end{verbatim}

Note the double brackets to indicate the list elements, i.e.~element one
is the vector of numbers and element two is a vector of a single
character.

We won't be working with lists in this workshop, but they are a flexible
way to store data of different types in R.

Accessing list elements uses double square brackets syntax, for example
\texttt{list\_1{[}{[}1{]}{]}} would return the first vector in our list.

And to access the first element in the first vector would combine double
and single square brackets like so:
\texttt{list\_1{[}{[}1{]}{]}{[}1{]}}.

Don't worry if you find this confusing, everyone does when they first
start with R.

\subsection{Matrices and arrays}\label{matrices-and-arrays}

Matrices store values in a two dimensional array, whilst arrays can have
n dimensions. We won't be using these either, but they are also valid R
objects.

\subsection{Data frames}\label{data-frames}

Data frames are two dimensional versions of lists, and this is form of
storing data we are going to be using. In a data frame each atomic
vector type becomes a column, and a data frame is formed by columns of
vectors of the same length. Each column element must be of the same
type, but the column types can vary.

Figure \ref{fig:df} shows an example data frame we'll refer to as saved
as the object \texttt{df} consisting of three rows and three columns.
Each column is a different atomic data type of the same length.



\begin{figure}

{\centering \includegraphics[width=0.8\linewidth]{img/data_frame} 

}

\caption{An example data frame \texttt{df}.}\label{fig:df}
\end{figure}

Packages in the tidyverse create a modified form of data frame called a
tibble. You can read about tibbles
\href{http://r4ds.had.co.nz/tibbles.html}{here}. One advantage of
tibbles is that they don't default to treating strings as factors. We
deal with modifying data frames when we work with our example data set.

Sub-setting data frames can also be done with square bracket syntax, but
as we have both rows and columns, we need to provide index values for
both row and column.

For example \texttt{df{[}1,2{]}} means \textbf{return the value of
\texttt{df} row 1, column 2}. This corresponds with the value
\texttt{A}.

We can also use the colon operator to choose several rows or columns,
and by leaving the row or column blank we return all rows or all
columns.

\begin{Shaded}
\begin{Highlighting}[]
\CommentTok{# Subset rows 1 and 2 of column 1}
\NormalTok{df[}\DecValTok{1}\OperatorTok{:}\DecValTok{2}\NormalTok{,}\DecValTok{1}\NormalTok{]}

\CommentTok{# Subset all rows of column 3}
\NormalTok{df[,}\DecValTok{3}\NormalTok{]}
\end{Highlighting}
\end{Shaded}

Again don't worry too much about this for now, we won't be doing to much
of this in this lesson, but it's important to be aware of the basic
syntax.

\section{Learning more R}\label{learning-more-r}

There are many places to start, but swirl can teach you interactively,
and at your own pace in RStudio.

Just follow the instructions via this link:
\url{http://swirlstats.com/students.html}

\emph{Hands-On Programming with R} by Garrett Grolemund is another great
resource for learning R.

Plus all the \href{https://www.tidyverse.org/learn/}{tidyverse links}.

\chapter{Creating scripts and importing data}\label{import}

Our analysis is of an example data set of observations for 7702 proteins
from cells in three control experiments and three treatment experiments.
The observations are signal intensity measurements from the mass
spectrometer. These intensities relate the concentration of protein
observed in each experiment and under each condition.

We consider raw data as the data as we receive it. This doesn't mean it
hasn't be processed in some way, it just means it hasn't been processed
by us. Generally speaking we don't change the raw data file, what we do
is import it and create an object in R which we then transform.

So let's understand how to import some data.

\section{Some definitions}\label{some-definitions}

\begin{itemize}
\tightlist
\item
  \textbf{Importing} means getting data into our R environment by
  creating an object that we can then manipulate. The raw data file
  remains unchanged.
\item
  \textbf{Inspecting} means looking at the dataset to understand what it
  contains.
\item
  \textbf{Tidying} refers to getting data into a consistent format that
  makes it easy to use in later steps.
\end{itemize}

\hypertarget{file-formats}{\subsection{Rectangular data and flat
formats}\label{file-formats}}

Two further things to note:

\begin{enumerate}
\def\labelenumi{\arabic{enumi}.}
\item
  Here we are only considering \textbf{rectangular data}, the sort that
  comes in rows and columns such as in a spreadsheet. Lots of our data
  types exist, such as images, but can also be handled by R. As
  mentioned in \ref{biocondutor} genomic data in particular has led to a
  project called \href{http://bioconductor.org/}{Bioconductor} for the
  development of analysis tools primarily in R, many of which deal with
  non-rectangular data, but this is beyond the scope here.
\item
  \textbf{Flat formats} are files that only contain plain text, with
  each line representing a set of observations and the variables
  separated by delimiters such as tabs, commas or spaces. Therefore
  there aren't multiple tables such as we'd get in an Excel file, or
  meta-data such as the colour highlighting of a cell in an Excel file.
  The advantages of flat files is that they can be opened and used by
  many different computing languages or programs. So unless there is a
  good reason not to use a flat format, and there are good reasons, they
  are the best way to store data in many situations.
\end{enumerate}

\section{Using scripts}\label{using-scripts}

Using the console is useful, but as we build up a workflow, that is to
say, writing code to:

\begin{itemize}
\tightlist
\item
  load packages
\item
  load data
\item
  explore the data
\item
  and output some results
\end{itemize}

Then it's much more useful to contain this in a script: a document of
our code.

Why? When we write and save our code in scripts, we can re-use it, share
it or edit it. But \textbf{most importantly a script is a record}.

Cmd/Ctrl + Shift + N will open a new script file up and you should see
something like Figure \ref{fig:script-pane} with the script editor pane
open:



\begin{figure}

{\centering \includegraphics[width=0.8\linewidth]{img/rstudio_screenshot_four_panes} 

}

\caption{Rstudio with the script editor pane open.}\label{fig:script-pane}
\end{figure}

\section{Running code}\label{running-code}

We can run a highlighted portion of code in your script if you click the
Run button at the top of the scripts pane as shown in Figure
\ref{fig:run-script}.



\begin{figure}

{\centering \includegraphics[width=0.8\linewidth]{img/run_script} 

}

\caption{Scripts can be run by clicking the Source button.}\label{fig:run-script}
\end{figure}

You can run the entire script by clicking the Source button.

Or we can run chunks of code if we split our script into sections, see
below.

\section{Creating a R script}\label{creating-a-r-script}

We first need to create a script that will form the basis of our
analysis.

Go to the file menu and select New Files \textgreater{} R script. This
should open the script editor pane.

Now let's save the script, by going to File \textgreater{} Save and we
should find ourselves prompted to save the script in our Project
Directory.

Following the advice about \protect\hyperlink{names}{naming things} we
can create a new R script called \texttt{01-bspr-workshop-july-2018}.

This name is machine readable (no spaces or special characters), human
readable, and works well with default ordering by beginning with
\texttt{01}.

\section{Setting up our environment}\label{setting-up-our-environment}

At the head of our script it's common to put a title, the name of the
author and the date, and any other useful information. This is created
as comments using the \texttt{\#} at the start of each line.

It's then usual to follow this by code to load the packages we need into
our our R environment using the \texttt{library()} function and
providing the name of the package we wish to load. Packages are
collections of R functions.

Often we break the code up into regions by adding dashes (or equals
symbols) to the comment line. This enables us to run chunks of the
script separately from running the whole script when using our code.

Here is a typical head for a script:

\begin{Shaded}
\begin{Highlighting}[]
\CommentTok{# My workshop script}
\CommentTok{# 7th July 2018}
\CommentTok{# Alistair Bailey}

\CommentTok{# Load packages ----------------------------------------------------------------}
\KeywordTok{library}\NormalTok{(plyr)}
\KeywordTok{library}\NormalTok{(tidyverse)}
\KeywordTok{library}\NormalTok{(gplots)}
\KeywordTok{library}\NormalTok{(pheatmap)}
\KeywordTok{library}\NormalTok{(gridExtra)}
\KeywordTok{library}\NormalTok{(VennDiagram)}
\KeywordTok{library}\NormalTok{(ggseqlogo)}
\end{Highlighting}
\end{Shaded}

\subsection{Bioconductor}\label{biocondutor}

As an aside there are many proteomics specific R packages, these are
generally found through
\href{https://www.bioconductor.org/}{Bioconductor} which is a project
that was initiated in 2001 to create tools for the analysis of
high-throughput genomic data, but also includes other 'omics data tools
\citep[\citet{huber2015}]{gentleman2004}.

Exploring Bioconductor is beyond our scope here, but well worth
exploring for manipulation and analysis of raw data formats such as
mzxml files.

\section{Importing data}\label{importing-data}

Assuming our data is in a \protect\hyperlink{file-formats}{flat format},
we can import it into our environment using the tidyverse \texttt{readr}
package.

If our data was an excel file, we can use the tidyverse \texttt{readxl}
package to import the data, but it will remove any meta-data and each
table in the excel file will become a separate R object as per tidy data
principles.

For the purposes of this workshop we have a \texttt{csv} (comma
separated variable) file.

If you haven't done so already Click here to download the example data
and save it to our project directory. Check the \texttt{Files} pane to
see it's there.

We then import data and assign it to an object we'll call \texttt{data}
like so:

\begin{Shaded}
\begin{Highlighting}[]
\CommentTok{# Import example data ----------------------------------------------------------}
\CommentTok{# Import the example data with read_csv from the readr package}
\NormalTok{dat <-}\StringTok{ }\NormalTok{readr}\OperatorTok{::}\KeywordTok{read_csv}\NormalTok{(}\StringTok{"data/070718-proteomics-example-data.csv"}\NormalTok{)}
\end{Highlighting}
\end{Shaded}

\begin{verbatim}
## Parsed with column specification:
## cols(
##   protein_accession = col_character(),
##   protein_description = col_character(),
##   control_1 = col_double(),
##   control_2 = col_double(),
##   control_3 = col_double(),
##   treatment_1 = col_double(),
##   treatment_2 = col_double(),
##   treatment_3 = col_double()
## )
\end{verbatim}

\section{Exploring the data}\label{exploring-the-data}

\subsection{\texorpdfstring{\texttt{glimpse}, \texttt{head} and
\texttt{str}}{glimpse, head and str}}\label{glimpse-head-and-str}

The first thing to do with any data set is to actually look at it. Here
are four ways to have look at the data in the \texttt{Console}: calling
the object directly, \texttt{glimpse}, \texttt{head} and \texttt{str}.

\begin{enumerate}
\def\labelenumi{\arabic{enumi}.}
\tightlist
\item
  We can just call the object and return it to the \texttt{Console},
  which may or may not be useful depending on the size and type of
  object we call.
\end{enumerate}

2 .\texttt{glimpse} is a tidyverse function that tries to show us as
much data in a data.frame or tibble as possible, telling us the
\protect\hyperlink{atomics}{atomic types} of data in the table, the
number of observations and the number of variables, and importantly
shows all the column variable names by transposing the table.

\begin{enumerate}
\def\labelenumi{\arabic{enumi}.}
\setcounter{enumi}{2}
\item
  \texttt{head} is a base function that shows us the 6 lines of a R
  object by default.
\item
  \texttt{str} is a base function that show the structure of a R object,
  so it provides a lot of information, but is not so easy to read.
\end{enumerate}

The outputs for these four functions is shown below:

\begin{Shaded}
\begin{Highlighting}[]
\CommentTok{# call object}
\NormalTok{dat}
\end{Highlighting}
\end{Shaded}

\begin{verbatim}
## # A tibble: 7,702 x 8
##    protein_accession protein_description     control_1 control_2 control_3
##    <chr>             <chr>                       <dbl>     <dbl>     <dbl>
##  1 VATA_HUMAN_P38606 V-type proton ATPase c~     0.811     0.858     1.04 
##  2 RL35A_HUMAN_P180~ 60S ribosomal protein ~     0.367     0.385     0.409
##  3 MYH10_HUMAN_P355~ Myosin-10 OS=Homo sapi~     2.98      4.62      2.87 
##  4 RHOG_HUMAN_P84095 Rho-related GTP-bindin~     0.142     0.224     0.128
##  5 PSA1_HUMAN_P25786 Proteasome subunit alp~     1.07      0.945     0.803
##  6 PRDX5_HUMAN_P300~ Peroxiredoxin-5_ mitoc~     0.566     0.540     0.488
##  7 ACLY_HUMAN_P53396 ATP-citrate synthase O~     5.00      4.22      5.03 
##  8 VDAC2_HUMAN_P458~ Voltage-dependent anio~     1.35      1.33      1.14 
##  9 LRC47_HUMAN_Q8N1~ Leucine-rich repeat-co~     0.927     0.770     1.17 
## 10 CH60_HUMAN_P10809 60 kDa heat shock prot~     9.45      8.41     10.4  
## # ... with 7,692 more rows, and 3 more variables: treatment_1 <dbl>,
## #   treatment_2 <dbl>, treatment_3 <dbl>
\end{verbatim}

\begin{Shaded}
\begin{Highlighting}[]
\CommentTok{# tidyverse glimpse function}
\KeywordTok{glimpse}\NormalTok{(dat)}
\end{Highlighting}
\end{Shaded}

\begin{verbatim}
## Observations: 7,702
## Variables: 8
## $ protein_accession   <chr> "VATA_HUMAN_P38606", "RL35A_HUMAN_P18077",...
## $ protein_description <chr> "V-type proton ATPase catalytic subunit A ...
## $ control_1           <dbl> 0.8114, 0.3672, 2.9815, 0.1424, 1.0748, 0....
## $ control_2           <dbl> 0.8575, 0.3853, 4.6176, 0.2238, 0.9451, 0....
## $ control_3           <dbl> 1.0381, 0.4091, 2.8709, 0.1281, 0.8032, 0....
## $ treatment_1         <dbl> 0.6448, 0.4109, 7.1670, 0.1643, 0.7884, 0....
## $ treatment_2         <dbl> 0.7190, 0.4634, 2.0052, 0.2466, 0.8798, 1....
## $ treatment_3         <dbl> 0.4805, 0.3561, 0.8995, 0.1268, 0.7631, 0....
\end{verbatim}

\begin{Shaded}
\begin{Highlighting}[]
\CommentTok{# head function}
\KeywordTok{head}\NormalTok{(dat)}
\end{Highlighting}
\end{Shaded}

\begin{verbatim}
## # A tibble: 6 x 8
##   protein_accession protein_description      control_1 control_2 control_3
##   <chr>             <chr>                        <dbl>     <dbl>     <dbl>
## 1 VATA_HUMAN_P38606 V-type proton ATPase ca~     0.811     0.858     1.04 
## 2 RL35A_HUMAN_P180~ 60S ribosomal protein L~     0.367     0.385     0.409
## 3 MYH10_HUMAN_P355~ Myosin-10 OS=Homo sapie~     2.98      4.62      2.87 
## 4 RHOG_HUMAN_P84095 Rho-related GTP-binding~     0.142     0.224     0.128
## 5 PSA1_HUMAN_P25786 Proteasome subunit alph~     1.07      0.945     0.803
## 6 PRDX5_HUMAN_P300~ Peroxiredoxin-5_ mitoch~     0.566     0.540     0.488
## # ... with 3 more variables: treatment_1 <dbl>, treatment_2 <dbl>,
## #   treatment_3 <dbl>
\end{verbatim}

\begin{Shaded}
\begin{Highlighting}[]
\CommentTok{# str function}
\KeywordTok{str}\NormalTok{(dat)}
\end{Highlighting}
\end{Shaded}

\begin{verbatim}
## Classes 'tbl_df', 'tbl' and 'data.frame':    7702 obs. of  8 variables:
##  $ protein_accession  : chr  "VATA_HUMAN_P38606" "RL35A_HUMAN_P18077" "MYH10_HUMAN_P35580" "RHOG_HUMAN_P84095" ...
##  $ protein_description: chr  "V-type proton ATPase catalytic subunit A OS=Homo sapiens GN=ATP6V1A PE=1 SV=2" "60S ribosomal protein L35a OS=Homo sapiens GN=RPL35A PE=1 SV=2" "Myosin-10 OS=Homo sapiens GN=MYH10 PE=1 SV=3" "Rho-related GTP-binding protein RhoG OS=Homo sapiens GN=RHOG PE=1 SV=1" ...
##  $ control_1          : num  0.811 0.367 2.982 0.142 1.075 ...
##  $ control_2          : num  0.858 0.385 4.618 0.224 0.945 ...
##  $ control_3          : num  1.038 0.409 2.871 0.128 0.803 ...
##  $ treatment_1        : num  0.645 0.411 7.167 0.164 0.788 ...
##  $ treatment_2        : num  0.719 0.463 2.005 0.247 0.88 ...
##  $ treatment_3        : num  0.48 0.356 0.899 0.127 0.763 ...
##  - attr(*, "spec")=List of 2
##   ..$ cols   :List of 8
##   .. ..$ protein_accession  : list()
##   .. .. ..- attr(*, "class")= chr  "collector_character" "collector"
##   .. ..$ protein_description: list()
##   .. .. ..- attr(*, "class")= chr  "collector_character" "collector"
##   .. ..$ control_1          : list()
##   .. .. ..- attr(*, "class")= chr  "collector_double" "collector"
##   .. ..$ control_2          : list()
##   .. .. ..- attr(*, "class")= chr  "collector_double" "collector"
##   .. ..$ control_3          : list()
##   .. .. ..- attr(*, "class")= chr  "collector_double" "collector"
##   .. ..$ treatment_1        : list()
##   .. .. ..- attr(*, "class")= chr  "collector_double" "collector"
##   .. ..$ treatment_2        : list()
##   .. .. ..- attr(*, "class")= chr  "collector_double" "collector"
##   .. ..$ treatment_3        : list()
##   .. .. ..- attr(*, "class")= chr  "collector_double" "collector"
##   ..$ default: list()
##   .. ..- attr(*, "class")= chr  "collector_guess" "collector"
##   ..- attr(*, "class")= chr "col_spec"
\end{verbatim}

To see the data in a \emph{spreadsheet} fashion use \texttt{View(dat)},
note the capital V and a new tab will open. This can also be launched
from the \texttt{Environment} tab by clicking on \texttt{dat}.

Although this provides us with some useful information, such as the
number of observations and variables, to understand more plotting the
data will be helpful as we'll see in Section \ref{normalisation}.

\subsection{Summary statisitics}\label{summary-statisitics}

Another useful way to quickly get a sense of the data is to use the
\texttt{summary} function, which will return summary of the spread of
the data and importantly if there are missing values. We can see
immediately below that the experimental replicates have different
distributions, and missing values that we need to deal with in Chapter
\ref{transform}.

\begin{Shaded}
\begin{Highlighting}[]
\KeywordTok{summary}\NormalTok{(dat)}
\end{Highlighting}
\end{Shaded}

\begin{verbatim}
##  protein_accession  protein_description   control_1        control_2     
##  Length:7702        Length:7702         Min.   : 0.001   Min.   : 0.000  
##  Class :character   Class :character    1st Qu.: 0.143   1st Qu.: 0.132  
##  Mode  :character   Mode  :character    Median : 0.345   Median : 0.322  
##                                         Mean   : 0.933   Mean   : 0.845  
##                                         3rd Qu.: 0.959   3rd Qu.: 0.845  
##                                         Max.   :31.944   Max.   :31.697  
##                                         NA's   :4888     NA's   :4828    
##    control_3       treatment_1      treatment_2      treatment_3    
##  Min.   : 0.001   Min.   : 0.000   Min.   : 0.002   Min.   : 0.002  
##  1st Qu.: 0.149   1st Qu.: 0.112   1st Qu.: 0.135   1st Qu.: 0.101  
##  Median : 0.388   Median : 0.286   Median : 0.319   Median : 0.254  
##  Mean   : 0.977   Mean   : 0.795   Mean   : 0.856   Mean   : 0.675  
##  3rd Qu.: 0.999   3rd Qu.: 0.780   3rd Qu.: 0.880   3rd Qu.: 0.682  
##  Max.   :31.320   Max.   :41.686   Max.   :28.234   Max.   :21.428  
##  NA's   :5087     NA's   :4739     NA's   :4902     NA's   :5074
\end{verbatim}

\chapter{\texorpdfstring{\texttt{dplyr} verbs and
piping}{dplyr verbs and piping}}\label{dplyr}

A core package in the tidyverse is \texttt{dplyr} for transforming data,
which is often used in conjunction with the \texttt{magrittr} package
that allows us to pipe multiple operations together.

The R4DS dplyr chapter is
\href{http://r4ds.had.co.nz/transform.html}{here} and for magrittr
\href{http://r4ds.had.co.nz/pipes.html}{here}.

The figures in this chapter we made for use with an ecological dataset
on rodent surveys, but the principles they illustrate are generic and
show the use of each function with or without the use of a pipe.

From R4DS:

``\emph{All \texttt{dplyr} verbs work similarly:}

\emph{1. The first argument is a data frame.}

\emph{2. The subsequent arguments describe what to do with the data
frame, using the variable names (without quotes).}

\emph{3. The result is a new data frame.}

\emph{Together these properties make it easy to chain together multiple
simple steps to achieve a complex result.}"

\section{Pipes}\label{pipes}

A pipe in R looks like this \texttt{\%\textgreater{}\%} and allows us to
send the output of one operation into another. This saves time and
space, and can make our code easier to read.

For example we can pipe the output of calling the \texttt{dat} object
into the \texttt{glimpse} function like so:

\begin{Shaded}
\begin{Highlighting}[]
\NormalTok{dat }\OperatorTok\StringTok{ }\KeywordTok{glimpse}\NormalTok{()}
\end{Highlighting}
\end{Shaded}

\begin{verbatim}
## Observations: 7,702
## Variables: 8
## $ protein_accession   <chr> "VATA_HUMAN_P38606", "RL35A_HUMAN_P18077",...
## $ protein_description <chr> "V-type proton ATPase catalytic subunit A ...
## $ control_1           <dbl> 0.8114, 0.3672, 2.9815, 0.1424, 1.0748, 0....
## $ control_2           <dbl> 0.8575, 0.3853, 4.6176, 0.2238, 0.9451, 0....
## $ control_3           <dbl> 1.0381, 0.4091, 2.8709, 0.1281, 0.8032, 0....
## $ treatment_1         <dbl> 0.6448, 0.4109, 7.1670, 0.1643, 0.7884, 0....
## $ treatment_2         <dbl> 0.7190, 0.4634, 2.0052, 0.2466, 0.8798, 1....
## $ treatment_3         <dbl> 0.4805, 0.3561, 0.8995, 0.1268, 0.7631, 0....
\end{verbatim}

This becomes even more useful when we combine pipes with \texttt{dplyr}
functions.

\section{Filter rows}\label{filter}

The \texttt{filter} function enables us to filter the rows of a data
frame according to a logical test (one that is \texttt{TRUE} or
\texttt{FALSE}). Here it filters rows in the surveys data where the year
variable is greater or equal to 1985.

\begin{center}\includegraphics[width=0.8\linewidth]{img/dplyr_filter} \end{center}

Let's try this with \texttt{dat} to filter the rows for proteins in
\texttt{control\_1} and \texttt{control\_2} experiments where the
observations are greater than 20:

\begin{Shaded}
\begin{Highlighting}[]
\NormalTok{dat }\OperatorTok\StringTok{ }\KeywordTok{filter}\NormalTok{(control_}\DecValTok{1} \OperatorTok{>}\StringTok{ }\DecValTok{20}\NormalTok{, control_}\DecValTok{2} \OperatorTok{>}\StringTok{ }\DecValTok{20}\NormalTok{)}
\end{Highlighting}
\end{Shaded}

\begin{verbatim}
## # A tibble: 2 x 8
##   protein_accession   protein_description    control_1 control_2 control_3
##   <chr>               <chr>                      <dbl>     <dbl>     <dbl>
## 1 MYH9_HUMAN_P35579   Myosin-9 OS=Homo sapi~      29.2      31.7      24.6
## 2 A0A087WWY3_HUMAN_A~ Filamin-A OS=Homo sap~      31.9      27.8      31.3
## # ... with 3 more variables: treatment_1 <dbl>, treatment_2 <dbl>,
## #   treatment_3 <dbl>
\end{verbatim}

Filtering is done with the following operators
\texttt{\textgreater{}},\texttt{\textless{}},\texttt{\textgreater{}=},\texttt{\textless{}=},\texttt{!=}
(not equal) and \texttt{==} for equal. Not the double equal sign.

\section{Arrange rows}\label{arrange-rows}

Arranging is similar to filter except that it changes the row order
according to the columns in ascending order. If you provide more than
one column name, each additional column will be used to break ties in
the values of preceding columns.

Here we arrange the surveys data according to the record identification
number.

\begin{center}\includegraphics[width=0.8\linewidth]{img/dplyr_arrange} \end{center}

To try that with \texttt{dat} let's arrange the data according to
\texttt{control\_1}:

\begin{Shaded}
\begin{Highlighting}[]
\NormalTok{dat }\OperatorTok\StringTok{ }\KeywordTok{arrange}\NormalTok{(control_}\DecValTok{1}\NormalTok{)}
\end{Highlighting}
\end{Shaded}

\begin{verbatim}
## # A tibble: 7,702 x 8
##    protein_accession protein_description     control_1 control_2 control_3
##    <chr>             <chr>                       <dbl>     <dbl>     <dbl>
##  1 PAL4G_HUMAN_P0DN~ Peptidyl-prolyl cis-tr~   0.001      0.0177    NA    
##  2 E5RGV5_HUMAN_E5R~ Nucleolysin TIA-1 isof~   0.0011    NA          0.093
##  3 E5RJP4_HUMAN_E5R~ Glutamine--fructose-6-~   0.002     NA         NA    
##  4 I3L3U1_HUMAN_I3L~ Myosin light chain 4 O~   0.00240   NA         NA    
##  5 ENPLL_HUMAN_Q58F~ Putative endoplasmin-l~   0.0026    NA         NA    
##  6 K1C15_HUMAN_P190~ Keratin_ type I cytosk~   0.00290    0.0615     0.122
##  7 B5ME44_HUMAN_B5M~ Outer dense fiber prot~   0.00290   NA         NA    
##  8 PANK3_HUMAN_Q9H9~ Pantothenate kinase 3 ~   0.0033    NA         NA    
##  9 RRS1_HUMAN_Q15050 Ribosome biogenesis re~   0.0035    NA         NA    
## 10 NFL_HUMAN_P07196  Neurofilament light po~   0.0035     0.315      0.564
## # ... with 7,692 more rows, and 3 more variables: treatment_1 <dbl>,
## #   treatment_2 <dbl>, treatment_3 <dbl>
\end{verbatim}

\section{Select columns}\label{select-columns}

Selecting is the verb we use to select columns of interest in the data.
Here we select only the \texttt{year} and \texttt{plot\_type} columns
and discard the rest.

\begin{center}\includegraphics[width=0.8\linewidth]{img/dplyr_select} \end{center}

Let's use select with \texttt{dat} to drop the protein description and
control experiments using negative indexing and keep everything else:

\begin{Shaded}
\begin{Highlighting}[]
\NormalTok{dat }\OperatorTok\StringTok{ }\KeywordTok{select}\NormalTok{(}\OperatorTok{-}\NormalTok{protein_description,}\OperatorTok{-}\NormalTok{(control_}\DecValTok{1}\OperatorTok{:}\NormalTok{control_}\DecValTok{3}\NormalTok{))}
\end{Highlighting}
\end{Shaded}

\begin{verbatim}
## # A tibble: 7,702 x 4
##    protein_accession  treatment_1 treatment_2 treatment_3
##    <chr>                    <dbl>       <dbl>       <dbl>
##  1 VATA_HUMAN_P38606        0.645       0.719       0.480
##  2 RL35A_HUMAN_P18077       0.411       0.463       0.356
##  3 MYH10_HUMAN_P35580       7.17        2.01        0.900
##  4 RHOG_HUMAN_P84095        0.164       0.247       0.127
##  5 PSA1_HUMAN_P25786        0.788       0.880       0.763
##  6 PRDX5_HUMAN_P30044       0.545       1.69        0.821
##  7 ACLY_HUMAN_P53396        4.67        5.01        3.57 
##  8 VDAC2_HUMAN_P45880       1.01        1.04        0.904
##  9 LRC47_HUMAN_Q8N1G4       1.22        1.01        0.593
## 10 CH60_HUMAN_P10809        8.31        8.31        5.73 
## # ... with 7,692 more rows
\end{verbatim}

\section{Create new variables}\label{mutate}

Creating new variables uses the \texttt{mutate} verb. Here I am creating
a new variable called \texttt{rodent\_type} that will create a new
column containing the type of rodent observed in each row.

\begin{center}\includegraphics[width=0.8\linewidth]{img/dplyr_mutate} \end{center}

Let's create a new variable for \texttt{dat} called \texttt{prot\_id}
that use the \texttt{str\_extract} function from the \texttt{stringr}
package to take the last 6 characters of the \texttt{protein\_accession}
variable, the \texttt{".\{6\}\$"} part is called a regular expression,
to keep just the UNIPROT id part of the string.

We'll use select to drop the other variables except the protein
accession afterwards via another pipe.

\begin{Shaded}
\begin{Highlighting}[]
\NormalTok{dat }\OperatorTok\StringTok{ }
\StringTok{  }\KeywordTok{mutate}\NormalTok{(}\DataTypeTok{prot_id =} \KeywordTok{str_extract}\NormalTok{(protein_accession,}\StringTok{".\{6\}$"}\NormalTok{)) }\OperatorTok\StringTok{ }
\StringTok{  }\KeywordTok{select}\NormalTok{(protein_accession, prot_id)}
\end{Highlighting}
\end{Shaded}

\begin{verbatim}
## # A tibble: 7,702 x 2
##    protein_accession  prot_id
##    <chr>              <chr>  
##  1 VATA_HUMAN_P38606  P38606 
##  2 RL35A_HUMAN_P18077 P18077 
##  3 MYH10_HUMAN_P35580 P35580 
##  4 RHOG_HUMAN_P84095  P84095 
##  5 PSA1_HUMAN_P25786  P25786 
##  6 PRDX5_HUMAN_P30044 P30044 
##  7 ACLY_HUMAN_P53396  P53396 
##  8 VDAC2_HUMAN_P45880 P45880 
##  9 LRC47_HUMAN_Q8N1G4 Q8N1G4 
## 10 CH60_HUMAN_P10809  P10809 
## # ... with 7,692 more rows
\end{verbatim}

\section{Create grouped summaries}\label{create-grouped-summaries}

The last key verb is \texttt{summarise} which collapses a data frame
into a single row.

For example, we could use it to find the average weight of all the
animals surveyed in the surveys data using \texttt{mean()}. (Here the
\texttt{na.rm\ =\ TRUE} argument is given to remove missing values from
the data, otherwise R would return \texttt{NA} when trying to average.)

\begin{center}\includegraphics[width=0.8\linewidth]{img/dplyr_summarise} \end{center}

\texttt{summarise} is most useful when paired with \texttt{group\_by}
which defines the variables upon which we operate upon.

Here if we group by \texttt{species\_id} and \texttt{rodent\_type}
together and then used \texttt{summarise} without any arguments we
return these two variables only.

\begin{center}\includegraphics[width=0.8\linewidth]{img/dplyr_group_by} \end{center}

We'll use the \texttt{mpg} dataset again to illustrate a grouped
summary. Here I'll group according fuel type \texttt{fl}, c = compressed
natural gas ,d = diesel, e = ethanol, p = premium and r = regular. Then
using summarise to calculate the mean highway (\texttt{hwy}) miles per
gallon, and the mean urban (\texttt{cty}) miles per gallon,the tables is
collapsed from 234 to five rows, one for each fuel type and two columns
for the mean mpg;s. This illustrates how grouped summaries provide a
very concise way of exploring data as we can immediately see the
relative fuel efficiences of each fuel type under two conditions.

\begin{Shaded}
\begin{Highlighting}[]
\CommentTok{# fl is fuel type. c = compressed natural gas ,d = diesel, }
\CommentTok{# e = ethanol, p = premium and r = regular.}
\NormalTok{mpg }\OperatorTok\StringTok{ }
\StringTok{  }\KeywordTok{group_by}\NormalTok{(fl) }\OperatorTok\StringTok{ }
\StringTok{  }\CommentTok{# Create summaries mean_hwy and mean_cty using the mean function, }
\StringTok{  }\CommentTok{# dropping any missing variables.}
\StringTok{  }\KeywordTok{summarise}\NormalTok{(}\DataTypeTok{mean_hwy =} \KeywordTok{mean}\NormalTok{(hwy, }\DataTypeTok{na.rm =}\NormalTok{ T), }\DataTypeTok{mean_cty =} \KeywordTok{mean}\NormalTok{(cty, }\DataTypeTok{na.rm =}\NormalTok{ T))}
\end{Highlighting}
\end{Shaded}

\begin{verbatim}
## # A tibble: 5 x 3
##   fl    mean_hwy mean_cty
##   <chr>    <dbl>    <dbl>
## 1 c         36      24   
## 2 d         33.6    25.6 
## 3 e         13.2     9.75
## 4 p         25.2    17.4 
## 5 r         23.0    16.7
\end{verbatim}

We'll use \texttt{dplyr} and pipes in Chapter \ref{transform}.

\chapter{Transforming and visualising proteomics data}\label{transform}

Having imported our data set of observations for 7702 proteins from
cells in three control experiments and three treatment experiments.
Remember, the observations are signal intensity measurements from the
mass spectrometer, and these intensities relate to the amount of protein
in each experiment and under each condition.

Now we will transform the data to examine the effect of the treatment on
the cellular proteome and visualise the output using a volcano plot and
a heatmap. The hypothesis we are testing is that treatment changes the
concentration of protein we observe.

A volcano plot is commonly used way of plotting changes in observed
values on the x-axis against the likelihood of observing that change due
to chance on the y-axis. Heatmaps are another way of visualising the
relative (increase and decrease of) amounts of observed values.

\section{Fold change and log-fold
change}\label{fold-change-and-log-fold-change}

Fold changes are ratios, the ratio of say protein expression before and
after treatment, where a value larger than 1 for a protein implies that
protein expression was greater after the treatment.

In life sciences, fold change is often reported as log-fold change. Why
is that? There are at least two reasons which can be shown by plotting.

One is that ratios are not symmetrical around 1, so it's difficult to
observe both changes in the forwards and backwards direcion
i.e.~proteins where expression went up and proteins where expression
went down due to treatment. When we transform ratios on a log scale, the
scale becomes symmetric around 0 and thus we can now observe the
distribution of ratios in terms of positive, negative or no change.




\begin{figure}
\centering
\includegraphics{bspr-workshop-2018_files/figure-latex/fold-change-1-1.pdf}
\caption{\label{fig:fold-change-1}Ratios are not symmetric around one, logratios are
symmetric around zero.}
\end{figure}

A second reason is that transforming values onto a log scale changes
where the numbers actually occur when plotted on that scale. If we
consider the log scale to represent magnitudes, then we can more easily
see changes of small and large magnitudes when we plot the data.

For example, a fold change of 32 times can be either a ratio 1/32 or
32/1.

As shown in Figure \ref{fig:fold-change-2}, 1/32 is much closer to 1
than 32/1, but transformed to a log scale we see that in terms of
magnitude of difference it is the same as 32/1.

Often the log transformation is to a base of 2 as each increment of 1
represents a doubling, but sometimes a base of 10 is used, for example
for p-values.



\begin{figure}
\centering
\includegraphics{bspr-workshop-2018_files/figure-latex/fold-change-2-1.pdf}
\caption{\label{fig:fold-change-2}Transformation of scales using log transformation.}
\end{figure}

\section{Dealing with missing values}\label{missing-values}

Unless we're really lucky, it's unlikely that we'll get observations for
the same numbers of proteins in all replicated experiments. This means
there will be missing values for some proteins when looking at all the
experiments together. This then raises the question of what to do about
the missing values? We have two choices:

\begin{enumerate}
\def\labelenumi{\arabic{enumi}.}
\tightlist
\item
  Only analyse the proteins that we have observations for in all
  experiments.
\item
  Impute values for the missing values from the existing observations.
\end{enumerate}

There are pros and cons to either approach. Here for simplicity we'll
use only the proteins for which we have observations in all assays.

We can drop the proteins with missing values by piping our data set to
the \texttt{drop\_na()} function from the \texttt{tidyr} package like
so. We assign this to a new object called \texttt{dat\_tidy}.

We'll use the summarise function to compare the number of proteins
before and after dropping the missing values using the \texttt{n()}
counting function.

\begin{Shaded}
\begin{Highlighting}[]
\CommentTok{# Remove the missing values}
\NormalTok{dat_tidy <-}\StringTok{ }\NormalTok{dat }\OperatorTok\StringTok{ }\KeywordTok{drop_na}\NormalTok{()}
\CommentTok{# Nunber of proteins in original data}
\NormalTok{dat }\OperatorTok\StringTok{ }\KeywordTok{summarise}\NormalTok{(}\DataTypeTok{Number_of_proteins =} \KeywordTok{n}\NormalTok{())}
\end{Highlighting}
\end{Shaded}

\begin{verbatim}
## # A tibble: 1 x 1
##   Number_of_proteins
##                <int>
## 1               7702
\end{verbatim}

\begin{Shaded}
\begin{Highlighting}[]
\CommentTok{# Nunber of proteins without missing values}
\NormalTok{dat_tidy }\OperatorTok\StringTok{ }\KeywordTok{summarise}\NormalTok{(}\DataTypeTok{Number_of_proteins =} \KeywordTok{n}\NormalTok{())}
\end{Highlighting}
\end{Shaded}

\begin{verbatim}
## # A tibble: 1 x 1
##   Number_of_proteins
##                <int>
## 1               1145
\end{verbatim}

This shrinks the dataset from 7,702 proteins to 1,145 proteins, so we
can see why imputing the missing values might be more atrractive.

One approach you might like to try is to impute the data by replacing
the missing values with the mean observation for each protein under each
condition.

\section{Data normalization}\label{normalisation}

To perform statistical inference, for example whether treatment
increases or decreases protein abundance, we need to account for the
variation that occurs from run to run on our spectrometers and each give
rise to a different distribution. This is as opposed to variation
arising from treatment versus control which we are interested in
understanding. Hence normalisation seeks to reduce the run-to-run
sources of variation.

A method of normalization introduced for DNA microarray analysis is
quantile normalisation \citep{bolstad2003}. There are various ways to
normalise data, so using quantile normalisation here is primarily to
demonstate the approach in R, you should consider what is best for your
data.

If we consider our proteomics data as a distribution of values, one
value for the concentration of each protein in our experiment that
together form a distribution. Figure \ref{fig:data-dist} shows the
distribution of protein concentrations observed for the three control
and three treatment assays. As we can see the distributions are
different for each assay.



\begin{figure}

{\centering \includegraphics[width=0.8\linewidth]{bspr-workshop-2018_files/figure-latex/data-dist-1} 

}

\caption{Protein data for six assays plotted as a distributions.}\label{fig:data-dist}
\end{figure}

A quantile represents a region of distribution, for example the 0.95
quantile is the value such that 95\% of the data lies below it. To
normalize two or more distributions with each other without recourse to
a reference distribution we:

\begin{enumerate}
\def\labelenumi{(\roman{enumi})}
\tightlist
\item
  Rank the value in each experiment (represented in the columns) from
  lowest to highest. In other words identify the quantiles for each
  protein in each experiment.
\item
  Sort each experiment (the columns) from lowest to highest value.
\item
  Calculate the mean across the rows for the sorted values.
\item
  Then substitute these mean values back according to rank for each
  experiment to restore the original order.
\end{enumerate}

This results in the highest ranking observation in each experiment
becoming the mean of the highest observations across all experiments,
the second ranking observation in each experiment becoming the mean of
the second highest observations across all experiments. Therefore the
distributions for each each experiment are now the same.

\href{https://davetang.org/muse/2014/07/07/quantile-normalisation-in-r/}{Dave
Tang's Blog:Quantile Normalisation in R} has more details on this
approach.





\begin{figure}

{\centering \includegraphics[width=0.8\linewidth]{img/quant_norm} 

}

\caption{Quantile Normalisation from
\href{https://twitter.com/rafalab/status/545586012219772928?ref_src=twsrc\%5Etfw}{Rafael
Irizarry's tweet}.}\label{fig:quant-norm}
\end{figure}

These result of quantile normalisation is that our distributions become
statisitcally identitical, which we can see by plotting the densities of
the normalized data. As shown in Figure \ref{fig:compare-normalisation}
the distributions all overlay.

We do this by creating a \protect\hyperlink{function-anatomy}{function}.
This takes a data frame as the arguement and pefrorms the steps
described to iterate through the data frame.

The code below is probably quite tricky to understand if you've not seen
\texttt{map} functions before, but they enable a function such as
\texttt{rank} or \texttt{sort} to be used on each column iteratively.
What's important here is to understand the aim, even if understanding
the code requires some more reading. You can read about
\href{http://r4ds.had.co.nz/iteration.html\#the-map-functions}{map
functions in R4DS}.

\begin{Shaded}
\begin{Highlighting}[]
\CommentTok{# Quantile normalisation : the aim is to give different distributions the}
\CommentTok{# same statistical properties}
\NormalTok{quantile_normalisation <-}\StringTok{ }\ControlFlowTok{function}\NormalTok{(df)\{}
  
  \CommentTok{# Find rank of values in each column}
\NormalTok{  df_rank <-}\StringTok{ }\KeywordTok{map_df}\NormalTok{(df,rank,}\DataTypeTok{ties.method=}\StringTok{"average"}\NormalTok{)}
  \CommentTok{# Sort observations in each column from lowest to highest }
\NormalTok{  df_sorted <-}\StringTok{ }\KeywordTok{map_df}\NormalTok{(df,sort)}
  \CommentTok{# Find row mean on sorted columns}
\NormalTok{  df_mean <-}\StringTok{ }\KeywordTok{rowMeans}\NormalTok{(df_sorted)}
  
  \CommentTok{# Function for substiting mean values according to rank }
\NormalTok{  index_to_mean <-}\StringTok{ }\ControlFlowTok{function}\NormalTok{(my_index, my_mean)\{}
    \KeywordTok{return}\NormalTok{(my_mean[my_index])}
\NormalTok{  \}}
  
  \CommentTok{# Replace value in each column with mean according to rank }
\NormalTok{  df_final <-}\StringTok{ }\KeywordTok{map_df}\NormalTok{(df_rank,index_to_mean, }\DataTypeTok{my_mean=}\NormalTok{df_mean)}
  
  \KeywordTok{return}\NormalTok{(df_final)}
\NormalTok{\}}
\end{Highlighting}
\end{Shaded}

The normalisation function is used by piping \texttt{dat\_tidy} first to
\texttt{select} to exclude the first two columns with the protein
accession and description in, and then to the normalisation function. We
re-bind the protein accession and description afterwards from
\texttt{dat\_tidy} by piping the output to \texttt{bind\_cols()}.

\begin{Shaded}
\begin{Highlighting}[]
\NormalTok{dat_norm <-}\StringTok{ }\NormalTok{dat_tidy }\OperatorTok\StringTok{ }\KeywordTok{select}\NormalTok{(}\OperatorTok{-}\KeywordTok{c}\NormalTok{(protein_accession}\OperatorTok{:}\NormalTok{protein_description)) }\OperatorTok\StringTok{ }
\StringTok{  }\KeywordTok{quantile_normalisation}\NormalTok{() }\OperatorTok\StringTok{ }
\StringTok{  }\KeywordTok{bind_cols}\NormalTok{(dat_tidy[,}\DecValTok{1}\OperatorTok{:}\DecValTok{2}\NormalTok{],.)}
\end{Highlighting}
\end{Shaded}




\begin{figure}

{\centering \includegraphics[width=0.8\linewidth]{bspr-workshop-2018_files/figure-latex/compare-normalisation-1} 

}

\caption{Comparison of the protein distributions before
normalization (left) and after quantile normalization (right).}\label{fig:compare-normalisation}
\end{figure}

\section{Hypothesis testing with the
t-test}\label{hypothesis-testing-with-the-t-test}

Having removed missing values and normalised the data, we can consider
our hypothesis: treatement changes the amount of protein we observe in
the cells.

In practice then, what we would like to know is whether the mean value
for each protein in our control and treatment assays differs due to
chance or due a real effect. We therefore need to calculate the
difference for each protein between treatment and control, and the
probability that any difference occurs due to chance. This is what the
p-value from the output of a t-test seeks to do. We need to perform 1145
t-tests.

\textbf{Note} There are biocondutor packages that contain functions
written to do this. However as a learning exercise we are going to work
through the problem.

Here I assume the reader is familiar with t-tests, but just to re-cap
some important points:

\begin{itemize}
\item
  We assume that the true population from which our data samples are
  indpendent, identically distributed and follow a normal distribution.
  This is not in fact true in practice, but t-test is robust to this
  assumption.
\item
  We assume unequal variances between the control and treatment for each
  protein. Hence we will perform a Welch's t-test for unequal variances.
\item
  We don't know whether the effect of the treatment is to increase or
  decrease the concentration of the protein, hence we will perform a
  two-sided t-test.
\item
  The observations for the proteins are for proteins of the same type
  but from independent experiments, rather than observations of the same
  individuals before and after treatment. Hence we test the observations
  as unpaired samples.
\end{itemize}

In R we use the base function \texttt{t.test} to perform Welch Two
Sample t-test and this outputs the p-values we need for each protein.
However, the challenge here is that our data has three observations for
each condition for each protein, hence we need to group the observations
for each protein according to the experimental condition as inputs to
each t-test.

We're going to follow what is called the \emph{split-apply-combine}
approach to deal with this problem:

\begin{enumerate}
\def\labelenumi{\arabic{enumi}.}
\tightlist
\item
  Split the data into control and treatment groups.
\item
  Apply the t-test function to each protein using the grouped inputs and
  store the p-value.
\item
  Combine all the p-values for each protein into a single vector.
\end{enumerate}

To this end I've created a function called \texttt{t\_test} that takes a
data frame and two group vectors as inputs. It splits the data into
\texttt{x} and \texttt{y} by subsetting the the data frame according to
the columns defined by the groups. The extra steps here are that the
subset data has to be unlisted and converted to numeric type for input
to the \texttt{t.test} function. We then perform the t-test, which will
calculate the mean of \texttt{x} and \texttt{y} and store the result in
a new object, and finally the function creates a data frame with a
single variable\texttt{p\_val} which is then returned as the function
output.

\begin{Shaded}
\begin{Highlighting}[]
\CommentTok{# T-test function for multiple experiments}
\NormalTok{t_test <-}\StringTok{ }\ControlFlowTok{function}\NormalTok{(dt,grp1,grp2)\{}
  \CommentTok{# Subset control group and convert to numeric}
\NormalTok{  x <-}\StringTok{ }\NormalTok{dt[grp1] }\OperatorTok\StringTok{ }\NormalTok{unlist }\OperatorTok\StringTok{ }\KeywordTok{as.numeric}\NormalTok{()}
  \CommentTok{# Subset treatment group and convert to numeric}
\NormalTok{  y <-}\StringTok{ }\NormalTok{dt[grp2] }\OperatorTok\StringTok{ }\NormalTok{unlist }\OperatorTok\StringTok{ }\KeywordTok{as.numeric}\NormalTok{()}
  \CommentTok{# Perform t-test using the mean of x and y}
\NormalTok{  result <-}\StringTok{ }\KeywordTok{t.test}\NormalTok{(x, y)}
  \CommentTok{# Extract p-values from the results}
\NormalTok{  p_vals <-}\StringTok{ }\KeywordTok{tibble}\NormalTok{(}\DataTypeTok{p_val =}\NormalTok{ result}\OperatorTok{$}\NormalTok{p.value)}
  \CommentTok{# Return p-values}
  \KeywordTok{return}\NormalTok{(p_vals)}
\NormalTok{\} }
\end{Highlighting}
\end{Shaded}

To use the \texttt{t\_test} function to perform many t-tests and not
just one t-test, we need to pass our \texttt{t\_test} function as an
arguement to another function.

This probably seems quite confusing, but the point here is that we want
to loop through every row in our table, and group the three control and
three treatment columns separately. Our \texttt{t\_test} function deals
with the latter problem, and by passing it to \texttt{adply} from the
\texttt{plyr} package we can loop through each row and it adds the
calculated p-values to our original table.

Concretely then, \texttt{adply} takes an array and applies the
\texttt{t\_test} function to each row and we supply the column group
indices arguments to the \texttt{t\_test} funcition. Here the indicies
are columns 3 to 5 for the control experiments and columns 6 to 8 for
the treatment functions. The function returns the input data with an
additional corresponding p-value column. \textbf{Note} I've piped the
output to \texttt{as.tibble()} to transform the data.frame output of
\texttt{adply} to tibble form to prevent errors that can occur if we try
to bind data frames and tibbles.

An important point here is that we can use this function for any number
of columns and rows providing our data is in the same tidy form by
changing the grouping indices.

\begin{Shaded}
\begin{Highlighting}[]
\CommentTok{# Apply t-test function to data using plyr adply}
\CommentTok{#  .margins = 1, slice by rows, .fun = t_test plus t_test arguements}
\NormalTok{dat_pvals <-}\StringTok{ }\NormalTok{plyr}\OperatorTok{::}\KeywordTok{adply}\NormalTok{(dat_norm,}\DataTypeTok{.margins =} \DecValTok{1}\NormalTok{, }\DataTypeTok{.fun =}\NormalTok{ t_test, }
                \DataTypeTok{grp1 =} \KeywordTok{c}\NormalTok{(}\DecValTok{3}\OperatorTok{:}\DecValTok{5}\NormalTok{), }\DataTypeTok{grp2 =} \KeywordTok{c}\NormalTok{(}\DecValTok{6}\OperatorTok{:}\DecValTok{8}\NormalTok{)) }\OperatorTok\StringTok{ }\KeywordTok{as.tibble}\NormalTok{()}
\end{Highlighting}
\end{Shaded}

To check our function, here's a comparision of calculating the first
protein p-value as a single t-test as shown in the following code and
the output of the function.

\begin{Shaded}
\begin{Highlighting}[]
\CommentTok{# Perform t-test on first protein}
\KeywordTok{t.test}\NormalTok{(}\KeywordTok{as.numeric}\NormalTok{(dat_norm[}\DecValTok{1}\NormalTok{,}\DecValTok{3}\OperatorTok{:}\DecValTok{5}\NormalTok{]),}
                    \KeywordTok{as.numeric}\NormalTok{(dat_norm[}\DecValTok{1}\NormalTok{,}\DecValTok{6}\OperatorTok{:}\DecValTok{8}\NormalTok{]))}\OperatorTok{$}\NormalTok{p.value}
\end{Highlighting}
\end{Shaded}

\begin{tabular}{rr}
\toprule
t\_test p-val & t.test p-val\\
\midrule
0.0927 & 0.0927\\
\bottomrule
\end{tabular}

We can plot a histogram of the p-values:

\begin{Shaded}
\begin{Highlighting}[]
\CommentTok{# Plot histogram}
\NormalTok{dat_pvals }\OperatorTok\StringTok{ }
\StringTok{  }\KeywordTok{ggplot}\NormalTok{(}\KeywordTok{aes}\NormalTok{(p_val)) }\OperatorTok{+}\StringTok{ }
\StringTok{  }\KeywordTok{geom_histogram}\NormalTok{(}\DataTypeTok{binwidth =} \FloatTok{0.05}\NormalTok{, }
           \DataTypeTok{boundary =} \FloatTok{0.5}\NormalTok{, }
           \DataTypeTok{fill =} \StringTok{"darkblue"}\NormalTok{,}
           \DataTypeTok{colour =} \StringTok{"white"}\NormalTok{) }\OperatorTok{+}
\StringTok{  }\KeywordTok{xlab}\NormalTok{(}\StringTok{"p-value"}\NormalTok{) }\OperatorTok{+}
\StringTok{  }\KeywordTok{ylab}\NormalTok{(}\StringTok{"Frequency"}\NormalTok{) }\OperatorTok{+}
\StringTok{  }\KeywordTok{theme_minimal}\NormalTok{()}
\end{Highlighting}
\end{Shaded}

\includegraphics{bspr-workshop-2018_files/figure-latex/p-val-plot-1.pdf}

\section{Calculating fold change}\label{calculating-fold-change}

To perform log transformation of the observations for each protein we
take our data and use select to exlude the columns of character vectors
and the pipe the output to \texttt{log2()} and use the pipe again to
create a data frame.

Then we use \texttt{bind\_cols} to bind the first two columns of
\texttt{dat\_pvals} followed by \texttt{dat\_log} and the last column of
\texttt{dat\_pvals}. This maintains the original column order.

\begin{Shaded}
\begin{Highlighting}[]
\CommentTok{# Select columns and log data}
\NormalTok{dat_log <-}\StringTok{ }\NormalTok{dat_pvals }\OperatorTok\StringTok{ }
\StringTok{  }\KeywordTok{select}\NormalTok{(}\OperatorTok{-}\KeywordTok{c}\NormalTok{(protein_accession,protein_description,p_val)) }\OperatorTok\StringTok{ }
\StringTok{  }\KeywordTok{log2}\NormalTok{()}

\CommentTok{# Bind columns to create transformed data frame}
\NormalTok{dat_combine <-}\StringTok{ }\KeywordTok{bind_cols}\NormalTok{(dat_pvals[,}\KeywordTok{c}\NormalTok{(}\DecValTok{1}\OperatorTok{:}\DecValTok{2}\NormalTok{)], dat_log, dat_pvals[,}\DecValTok{9}\NormalTok{]) }
\end{Highlighting}
\end{Shaded}

The log fold change is then the difference between the log mean control
and log mean treatment values. By use of grouping by the protein
accession we can then use \texttt{mutate} to create new variables that
calculate the mean values and then calculate the \texttt{log\_fc}.
Whilst we're about it, we can also calculate a -log10(p-value). As with
fold change, transforming the p-value on a log10 scale means that a
p-value of 0.05 or below is transformed to 1.3 or above and a p-value of
0.01 is equal to 2.

\begin{Shaded}
\begin{Highlighting}[]
\NormalTok{dat_fc <-}\StringTok{ }\NormalTok{dat_combine }\OperatorTok\StringTok{ }
\StringTok{  }\KeywordTok{group_by}\NormalTok{(protein_accession) }\OperatorTok\StringTok{ }
\StringTok{  }\KeywordTok{mutate}\NormalTok{(}\DataTypeTok{mean_control =} \KeywordTok{mean}\NormalTok{(}\KeywordTok{c}\NormalTok{(control_}\DecValTok{1}\NormalTok{,}
\NormalTok{                               control_}\DecValTok{2}\NormalTok{,}
\NormalTok{                               control_}\DecValTok{3}\NormalTok{)),}
                             \DataTypeTok{mean_treatment=} \KeywordTok{mean}\NormalTok{(}\KeywordTok{c}\NormalTok{(treatment_}\DecValTok{1}\NormalTok{,}
\NormalTok{                                                    treatment_}\DecValTok{2}\NormalTok{,}
\NormalTok{                                                    treatment_}\DecValTok{3}\NormalTok{)),}
         \DataTypeTok{log_fc =}\NormalTok{ mean_control }\OperatorTok{-}\StringTok{ }\NormalTok{mean_treatment,}
         \DataTypeTok{log_pval =} \OperatorTok{-}\DecValTok{1}\OperatorTok{*}\KeywordTok{log10}\NormalTok{(p_val))}
\end{Highlighting}
\end{Shaded}

The next step is not necessary, but for ease of viewing we subset
\texttt{dat\_fc} to create a new data frame called \texttt{dat\_tf} that
contains only four variables. We could potentially write this to a csv
file for sharing.

\begin{Shaded}
\begin{Highlighting}[]
\CommentTok{# Final transformed data}
\NormalTok{dat_tf <-}\StringTok{ }\NormalTok{dat_fc }\OperatorTok\StringTok{ }\KeywordTok{select}\NormalTok{(protein_accession,}
\NormalTok{                            protein_description,}
\NormalTok{                            log_fc, log_pval)}
\end{Highlighting}
\end{Shaded}

Let's look at the head of the final table:

\begin{tabular}{llrr}
\toprule
protein\_accession & protein\_description & log\_fc & log\_pval\\
\midrule
VATA\_HUMAN\_P38606 & V-type proton ATPase catalytic subunit A OS=Homo sapiens GN=ATP6V1A PE=1 SV=2 & 0.3687886 & 1.0327506\\
RL35A\_HUMAN\_P18077 & 60S ribosomal protein L35a OS=Homo sapiens GN=RPL35A PE=1 SV=2 & -0.2505780 & 1.1400318\\
MYH10\_HUMAN\_P35580 & Myosin-10 OS=Homo sapiens GN=MYH10 PE=1 SV=3 & 0.3838733 & 0.0056494\\
RHOG\_HUMAN\_P84095 & Rho-related GTP-binding protein RhoG OS=Homo sapiens GN=RHOG PE=1 SV=1 & -0.3417452 & 0.3775483\\
PSA1\_HUMAN\_P25786 & Proteasome subunit alpha type-1 OS=Homo sapiens GN=PSMA1 PE=1 SV=1 & 0.0371316 & 0.0920101\\
\bottomrule
\end{tabular}

\section{Visualising the transformed
data}\label{visualising-the-transformed-data}

Plotting a histogram of the log fold change gives an indication of
whether the treatment has an effect on the cells. Most values are close
to zero, but there are some observations far above and below zero
suggesting the treatment does have an effect.



\begin{Shaded}
\begin{Highlighting}[]
\CommentTok{# Plot a histogram to look at the distribution.}
\NormalTok{dat_tf }\OperatorTok
\StringTok{  }\KeywordTok{ggplot}\NormalTok{(}\KeywordTok{aes}\NormalTok{(log_fc)) }\OperatorTok{+}\StringTok{ }
\StringTok{  }\KeywordTok{geom_histogram}\NormalTok{(}\DataTypeTok{binwidth =} \FloatTok{0.5}\NormalTok{,}
                 \DataTypeTok{boundary =} \FloatTok{0.5}\NormalTok{,}
           \DataTypeTok{fill =} \StringTok{"darkblue"}\NormalTok{,}
           \DataTypeTok{colour =} \StringTok{"white"}\NormalTok{) }\OperatorTok{+}
\StringTok{  }\KeywordTok{xlab}\NormalTok{(}\StringTok{"log2 fold change"}\NormalTok{) }\OperatorTok{+}
\StringTok{  }\KeywordTok{ylab}\NormalTok{(}\StringTok{"Frequency"}\NormalTok{) }\OperatorTok{+}
\StringTok{  }\KeywordTok{theme_minimal}\NormalTok{()}
\end{Highlighting}
\end{Shaded}

\begin{figure}

{\centering \includegraphics[width=0.8\linewidth]{bspr-workshop-2018_files/figure-latex/log-fc-1} 

}

\caption{Histogram of log fold change.}\label{fig:log-fc}
\end{figure}

However, we don't know if these fold changes are dueto chance or not,
which is why we calculated the p-values. A volcano plot will include the
p-value information.

\section{Volcano plot}\label{volcano-plot}

A volcano plot is a plot of the log fold change in the observation
between two conditions on the x-axis, for example the protein expression
between treatment and control conditions. On the y-axis is the
corresponding p-value for each observation, representing the likelihood
that an observed change is due to the different conditions rather than
arising from a natural variation in the fold change that might be
observed if we performed many replications of the experiment.

The aim of a volcano plot is to enable the viewer to quickly see the
effect (if any) of an experiment with two conditions on many species
(i.e.~proteins) in terms of both an increase and decrease of the
observed value.

Like all plots it has it's good and bad points, namely it's good that we
can visualise a lot of complex information in one plot. However this is
also it's main weakness, it's rather complicated to understand in one
glance.

\begin{Shaded}
\begin{Highlighting}[]
\NormalTok{dat_tf }\OperatorTok\StringTok{ }\KeywordTok{ggplot}\NormalTok{(}\KeywordTok{aes}\NormalTok{(log_fc,log_pval)) }\OperatorTok{+}\StringTok{ }\KeywordTok{geom_point}\NormalTok{()}
\end{Highlighting}
\end{Shaded}

\includegraphics{bspr-workshop-2018_files/figure-latex/volcano-plot-1.pdf}

However it would be much more useful with some extra formatting, so the
code below shows one way to transform the data to include a threshold
which can then be used by ggplot to create an additional aesthetic. The
code below also includes some extra formatiing which the reader can
explore.




\begin{Shaded}
\begin{Highlighting}[]
\NormalTok{dat_tf }\OperatorTok
\StringTok{  }\CommentTok{# Add a threhold for significant observations}
\StringTok{  }\KeywordTok{mutate}\NormalTok{(}\DataTypeTok{threshold =} \KeywordTok{if_else}\NormalTok{(log_fc }\OperatorTok{>=}\StringTok{ }\DecValTok{2} \OperatorTok{&}\StringTok{ }\NormalTok{log_pval }\OperatorTok{>=}\StringTok{ }\FloatTok{1.3} \OperatorTok{|}
\StringTok{                               }\NormalTok{log_fc }\OperatorTok{<=}\StringTok{ }\OperatorTok{-}\DecValTok{2} \OperatorTok{&}\StringTok{ }\NormalTok{log_pval }\OperatorTok{>=}\StringTok{ }\FloatTok{1.3}\NormalTok{,}\StringTok{"A"}\NormalTok{, }\StringTok{"B"}\NormalTok{)) }\OperatorTok
\StringTok{  }\CommentTok{# Plot with points coloured according to the threshold}
\StringTok{  }\KeywordTok{ggplot}\NormalTok{(}\KeywordTok{aes}\NormalTok{(log_fc,log_pval, }\DataTypeTok{colour =}\NormalTok{ threshold)) }\OperatorTok{+}
\StringTok{  }\KeywordTok{geom_point}\NormalTok{(}\DataTypeTok{alpha =} \FloatTok{0.5}\NormalTok{) }\OperatorTok{+}\StringTok{ }\CommentTok{# Alpha sets the transparency of the points}
\StringTok{  }\CommentTok{# Add dotted lines to indicate the threshold, semi-transparent}
\StringTok{  }\KeywordTok{geom_hline}\NormalTok{(}\DataTypeTok{yintercept =} \FloatTok{1.3}\NormalTok{, }\DataTypeTok{linetype =} \DecValTok{2}\NormalTok{, }\DataTypeTok{alpha =} \FloatTok{0.5}\NormalTok{) }\OperatorTok{+}\StringTok{ }
\StringTok{  }\KeywordTok{geom_vline}\NormalTok{(}\DataTypeTok{xintercept =} \DecValTok{2}\NormalTok{, }\DataTypeTok{linetype =} \DecValTok{2}\NormalTok{, }\DataTypeTok{alpha =} \FloatTok{0.5}\NormalTok{) }\OperatorTok{+}
\StringTok{  }\KeywordTok{geom_vline}\NormalTok{(}\DataTypeTok{xintercept =} \OperatorTok{-}\DecValTok{2}\NormalTok{, }\DataTypeTok{linetype =} \DecValTok{2}\NormalTok{, }\DataTypeTok{alpha =} \FloatTok{0.5}\NormalTok{) }\OperatorTok{+}
\StringTok{  }\CommentTok{# Set the colour of the points}
\StringTok{  }\KeywordTok{scale_colour_manual}\NormalTok{(}\DataTypeTok{values =} \KeywordTok{c}\NormalTok{(}\StringTok{"A"}\NormalTok{=}\StringTok{ "red"}\NormalTok{, }\StringTok{"B"}\NormalTok{=}\StringTok{ "black"}\NormalTok{)) }\OperatorTok{+}
\StringTok{  }\KeywordTok{xlab}\NormalTok{(}\StringTok{"log2 fold change"}\NormalTok{) }\OperatorTok{+}\StringTok{ }\KeywordTok{ylab}\NormalTok{(}\StringTok{"-log10 p-value"}\NormalTok{) }\OperatorTok{+}\StringTok{ }\CommentTok{# Relabel the axes}
\StringTok{  }\KeywordTok{theme_minimal}\NormalTok{() }\OperatorTok{+}\StringTok{ }\CommentTok{# Set the theme}
\StringTok{  }\KeywordTok{theme}\NormalTok{(}\DataTypeTok{legend.position=}\StringTok{"none"}\NormalTok{) }\CommentTok{# Hide the legend}
\end{Highlighting}
\end{Shaded}

\begin{figure}

{\centering \includegraphics[width=0.8\linewidth]{bspr-workshop-2018_files/figure-latex/nice-vplot-1} 

}

\caption{A volcano plot with formatting to highlight the significant
proteins}\label{fig:nice-vplot}
\end{figure}

\subsection{But which proteins are the significant
observations?}\label{but-which-proteins-are-the-significant-observations}

To extract the proteins in red in Figure \ref{fig:nice-vplot} we filter
\texttt{dat\_tf} according to our threshold and then create a new
variable using the \texttt{str\_extract} function used in Section
\ref{mutate}.

\textbf{Note} We need to ungroup the data we grouped when calculating
the log\_fc to be able to select columns without keeping the grouping
variable column too.

\begin{Shaded}
\begin{Highlighting}[]
\NormalTok{dat_tf }\OperatorTok
\StringTok{  }\CommentTok{# Filter for significant observations}
\StringTok{  }\KeywordTok{filter}\NormalTok{(log_pval }\OperatorTok{>=}\StringTok{ }\FloatTok{1.3} \OperatorTok{&}\StringTok{ }\NormalTok{(log_fc }\OperatorTok{>=}\StringTok{ }\DecValTok{2} \OperatorTok{|}\StringTok{ }\NormalTok{log_fc }\OperatorTok{<=}\StringTok{ }\OperatorTok{-}\DecValTok{2}\NormalTok{)) }\OperatorTok\StringTok{ }
\StringTok{  }\CommentTok{# Get last six characters}
\StringTok{  }\KeywordTok{mutate}\NormalTok{(}\DataTypeTok{prot_id =} \KeywordTok{str_extract}\NormalTok{(protein_accession,}\StringTok{".\{6\}$"}\NormalTok{)) }\OperatorTok\StringTok{ }
\StringTok{  }\CommentTok{# Ungroup the data}
\StringTok{  }\KeywordTok{ungroup}\NormalTok{() }\OperatorTok\StringTok{ }
\StringTok{  }\CommentTok{# Select columns of interest}
\StringTok{  }\KeywordTok{select}\NormalTok{(prot_id,protein_description,log_fc,log_pval)}
\end{Highlighting}
\end{Shaded}

\begin{verbatim}
## # A tibble: 5 x 4
##   prot_id protein_description                              log_fc log_pval
##   <chr>   <chr>                                             <dbl>    <dbl>
## 1 Q02952  A-kinase anchor protein 12 OS=Homo sapiens GN=A~  -2.29     1.52
## 2 O94808  Glutamine--fructose-6-phosphate aminotransferas~  -3.09     2.45
## 3 H7BYV1  Interferon-induced transmembrane protein 2 (Fra~   2.05     2.46
## 4 P06756  Integrin alpha-V OS=Homo sapiens GN=ITGAV PE=1 ~  -2.22     2.62
## 5 Q8TDI0  Chromodomain-helicase-DNA-binding protein 5 OS=~   5.04     1.60
\end{verbatim}

\section{Creating a heatmap}\label{creating-a-heatmap}

Here we'll create a heatmap using the \texttt{heatmap.2} function from
the \texttt{gplots} package and the \texttt{pheatmap} function from the
\texttt{pheatmap} package.

To create a heatmap we need to perform a few more transformations:

\begin{enumerate}
\def\labelenumi{\arabic{enumi}.}
\item
  Filter the data according to a threshold of significance. This time
  we'll use a more relaxed log\_fc cut-off to ensure we have enough
  proteins to plot. At the same time we'll extract the protein ids as
  before.
\item
  We then have to transform our filtered data into a
  \texttt{matrix.data.frame} object for use with \texttt{pheatmap}. We
  name the rows with the protein ids
\item
  We'll use base R function \texttt{scale} to centre our log transformed
  data around zero. To do this per experiment we transpose the matrix as
  scale centres rows, and the flip the matrix back again.
\end{enumerate}

\begin{Shaded}
\begin{Highlighting}[]
\CommentTok{# Keep the same p-val cut-off, but relax the log_fc to 1 which represents a }
\CommentTok{# doubling}
\NormalTok{dat_filt <-}\StringTok{ }\NormalTok{dat_fc }\OperatorTok
\StringTok{  }\KeywordTok{filter}\NormalTok{(log_pval }\OperatorTok{>=}\StringTok{ }\FloatTok{1.3} \OperatorTok{&}\StringTok{ }\NormalTok{(log_fc }\OperatorTok{>=}\StringTok{ }\DecValTok{1} \OperatorTok{|}\StringTok{ }\NormalTok{log_fc }\OperatorTok{<=}\StringTok{ }\OperatorTok{-}\DecValTok{1}\NormalTok{)) }\OperatorTok\StringTok{ }
\StringTok{  }\KeywordTok{mutate}\NormalTok{(}\DataTypeTok{prot_id =} \KeywordTok{str_extract}\NormalTok{(protein_accession,}\StringTok{".\{6\}$"}\NormalTok{))}

\CommentTok{# Convert to matrix data frame}
\NormalTok{dat_matrix <-}\StringTok{ }\KeywordTok{as.matrix.data.frame}\NormalTok{(dat_filt[,}\DecValTok{3}\OperatorTok{:}\DecValTok{8}\NormalTok{]) }
\CommentTok{# Name the rows with protein ids}
\KeywordTok{row.names}\NormalTok{(dat_matrix) <-}\StringTok{ }\NormalTok{dat_filt}\OperatorTok{$}\NormalTok{prot_id}
\CommentTok{# Transpose and scale the data to a mean of zero and sd of one}
\NormalTok{dat_scaled <-}\StringTok{ }\KeywordTok{scale}\NormalTok{(}\KeywordTok{t}\NormalTok{(dat_matrix)) }\OperatorTok\StringTok{ }\KeywordTok{t}\NormalTok{()}
\end{Highlighting}
\end{Shaded}

\subsection{Calculating similarity and
clustering}\label{calculating-similarity-and-clustering}

At this point we could just plot the data, but to understand what the
heatmap functions do to cluster the data, let's step through the
process.

Our data here as log fold change in concentrations, but how do we group
them? The simplest thing to do is to turn the data into distances, as a
measure of similarity, where close things are similar and distant things
are dissimilar.

The Euclidean distance \(d\) between a pair of observations \(x_i\) and
\(y_i\) is defined as:

\(d = \sqrt{\sum{_i}(x_i - y_i)^2}\)

Lets calculate the distance between the columns in \texttt{dat\_scaled}.

In \texttt{dat\_scaled} the experiments are in the columns. In
calculating the distance is between the experiments for all the proteins
in each experiment. What would we expect?

We'd expect the controls to be close to each other and the treated to be
close to each other, right?

Let's do this in detail, for example the distance between
\texttt{control\_1} and \texttt{control\_2} is
\texttt{sqrt(sum((dat\_scaled{[},1{]}\ -\ dat\_scaled{[},2{]})\^{}2))}.

This means we take the column 2 values from column 1 values, squaring
the results and summing them all to a single value and taking the square
root to find the linear distance between these rows, which is 3.26.

You can check this against the first value in \texttt{d1} that we
calculate below in using \texttt{dist}.

We do the same for the proteins, but we don't know what to expect.
Here's the code for calculating both distance matrices

\begin{Shaded}
\begin{Highlighting}[]
\CommentTok{# Transpose the matrix to calculate distance between experiments, row-wise}
\NormalTok{d1 <-}\StringTok{ }\NormalTok{dat_scaled }\OperatorTok\StringTok{ }\KeywordTok{t}\NormalTok{() }\OperatorTok
\StringTok{  }\KeywordTok{dist}\NormalTok{(.,}\DataTypeTok{method =} \StringTok{"euclidean"}\NormalTok{, }\DataTypeTok{diag =} \OtherTok{FALSE}\NormalTok{, }\DataTypeTok{upper =} \OtherTok{FALSE}\NormalTok{)}
\CommentTok{# Calculate the distance between proteins row-wise }
\NormalTok{d2 <-}\StringTok{ }\NormalTok{dat_scaled }\OperatorTok
\StringTok{  }\KeywordTok{dist}\NormalTok{(.,}\DataTypeTok{method =} \StringTok{"euclidean"}\NormalTok{, }\DataTypeTok{diag =} \OtherTok{FALSE}\NormalTok{, }\DataTypeTok{upper =} \OtherTok{FALSE}\NormalTok{)}

\CommentTok{# Show the values for d1}
\KeywordTok{round}\NormalTok{(d1,}\DecValTok{2}\NormalTok{)}
\end{Highlighting}
\end{Shaded}

\begin{verbatim}
##             control_1 control_2 control_3 treatment_1 treatment_2
## control_2        3.26                                            
## control_3        3.20      3.27                                  
## treatment_1      8.97      8.60      8.65                        
## treatment_2      9.40      8.98      8.86        2.35            
## treatment_3      9.04      8.56      8.50        2.46        1.71
\end{verbatim}

Having calculated the distance matrices, we can cluster proteins and
experiments accordingly.

There are lots of flavours of clustering, and no clear way to say which
is best. Here we'll use the Ward criterion for clustering which attempts
to minimise the variance within clusters as it merges the data into
clusters, using the distances we've calculated. The data is merged from
the bottom up (aka agglomeration) adding data points to a cluster and
splitting them according to the variance criterion.

See Wikipedia for more detail:
\href{https://en.wikipedia.org/wiki/Hierarchical_clustering}{Hierarchical
clustering}

\begin{Shaded}
\begin{Highlighting}[]
\CommentTok{# Clustering distance between experiments using Ward linkage}
\NormalTok{c1 <-}\StringTok{ }\KeywordTok{hclust}\NormalTok{(d1, }\DataTypeTok{method =} \StringTok{"ward.D2"}\NormalTok{, }\DataTypeTok{members =} \OtherTok{NULL}\NormalTok{)}
\CommentTok{# Clustering distance between proteins using Ward linkage}
\NormalTok{c2 <-}\StringTok{ }\KeywordTok{hclust}\NormalTok{(d2, }\DataTypeTok{method =} \StringTok{"ward.D2"}\NormalTok{, }\DataTypeTok{members =} \OtherTok{NULL}\NormalTok{)}
\end{Highlighting}
\end{Shaded}

Now lets look at the dendrograms made by clustering our distance
matrices \texttt{d1} and \texttt{d2}:



\begin{Shaded}
\begin{Highlighting}[]
\CommentTok{# Check clustering by plotting dendrograms}
\KeywordTok{par}\NormalTok{(}\DataTypeTok{mfrow=}\KeywordTok{c}\NormalTok{(}\DecValTok{2}\NormalTok{,}\DecValTok{1}\NormalTok{),}\DataTypeTok{cex=}\FloatTok{0.5}\NormalTok{) }\CommentTok{# Make 2 rows, 1 col plot frame and shrink labels}
\KeywordTok{plot}\NormalTok{(c1); }\KeywordTok{plot}\NormalTok{(c2) }\CommentTok{# Plot both cluster dendrograms}
\end{Highlighting}
\end{Shaded}

\begin{figure}

{\centering \includegraphics[width=0.8\linewidth]{bspr-workshop-2018_files/figure-latex/dendrograms-1} 

}

\caption{Dendrograms of Ward clustering of distance matrices}\label{fig:dendrograms}
\end{figure}

As we'd expect, Figure \ref{fig:dendrograms} shows the controls and
treatments cluster respectively.

\subsection{Plotting the heatmap}\label{plotting-the-heatmap}

The \texttt{heatmap.2} function from the \texttt{gplots} package will
automatically perform the distance calculation and clustering we
performed, and it can also do the scaling we did. It only requires the
matrix as an input by default. It will use a different clustering method
by default.

However, as we've performed scaling and calculated the clusters, we can
pass them to heatmap function.

I'll leave it to the reader to explore all the options here, but the
concept in the code below to create Figure \ref{fig:heatmap2} is:

\begin{itemize}
\tightlist
\item
  Create a 25 increment blue/white/red colour pallette
\item
  Pipe \texttt{dat\_scaled} to a function that renames the colums
\item
  Pipe this to the \texttt{heatmap.2} function
\item
  Pass the clusters \texttt{c1} and \texttt{c2} to the plot
\item
  Change some aesthetics such as the colours, and the font sizes
\end{itemize}




\begin{Shaded}
\begin{Highlighting}[]
\CommentTok{# Set colours for heatmap, 25 increments}
\NormalTok{my_palette <-}\StringTok{ }\KeywordTok{colorRampPalette}\NormalTok{(}\KeywordTok{c}\NormalTok{(}\StringTok{"blue"}\NormalTok{,}\StringTok{"white"}\NormalTok{,}\StringTok{"red"}\NormalTok{))(}\DataTypeTok{n =} \DecValTok{25}\NormalTok{)}

\CommentTok{# Plot heatmap with heatmap.2}
\KeywordTok{par}\NormalTok{(}\DataTypeTok{cex.main=}\FloatTok{0.75}\NormalTok{) }\CommentTok{# Shrink title fonts on plot}
\NormalTok{dat_scaled }\OperatorTok\StringTok{ }
\StringTok{  }\CommentTok{# Rename the comlums}
\StringTok{  }\NormalTok{magrittr}\OperatorTok{::}\KeywordTok{set_colnames}\NormalTok{(}\KeywordTok{c}\NormalTok{(}\StringTok{"Ctl 1"}\NormalTok{, }\StringTok{"Ctl 2"}\NormalTok{, }\StringTok{"Ctl 3"}\NormalTok{,}
                                    \StringTok{"Trt 1"}\NormalTok{, }\StringTok{"Trt 2"}\NormalTok{, }\StringTok{"Trt 3"}\NormalTok{)) }\OperatorTok\StringTok{ }
\StringTok{  }\CommentTok{# Plot heatmap}
\StringTok{  }\NormalTok{gplots}\OperatorTok{::}\KeywordTok{heatmap.2}\NormalTok{(.,                     }\CommentTok{# Tidy, normalised data}
          \DataTypeTok{Colv=}\KeywordTok{as.dendrogram}\NormalTok{(c1),     }\CommentTok{# Experiments clusters in cols}
          \DataTypeTok{Rowv=}\KeywordTok{as.dendrogram}\NormalTok{(c2),     }\CommentTok{# Protein clusters in rows}
          \DataTypeTok{revC=}\OtherTok{TRUE}\NormalTok{,                  }\CommentTok{# Flip plot to match pheatmap}
          \DataTypeTok{density.info=}\StringTok{"histogram"}\NormalTok{,   }\CommentTok{# Plot histogram of data and colour key}
          \DataTypeTok{trace=}\StringTok{"none"}\NormalTok{,               }\CommentTok{# Turn of trace lines from heat map}
          \DataTypeTok{col =}\NormalTok{ my_palette,           }\CommentTok{# Use my colour scheme}
          \DataTypeTok{cexRow=}\FloatTok{0.6}\NormalTok{,}\DataTypeTok{cexCol=}\FloatTok{0.75}\NormalTok{)     }\CommentTok{# Amend row and column label fonts}
\end{Highlighting}
\end{Shaded}

\begin{figure}

{\centering \includegraphics[width=0.8\linewidth]{bspr-workshop-2018_files/figure-latex/heatmap2-1} 

}

\caption{Heatmap created with \texttt{heatmap.2} using the
clusters calculated.}\label{fig:heatmap2}
\end{figure}

An alternative and more \texttt{ggplot} style is to use the
\texttt{pheatmap} package and function \citep{R-pheatmap}.

In Figure \ref{fig:pheatmap} \texttt{dat\_scaled} is piped to
\texttt{set\_columns} again to rename the experiments for aesthetic
reasons. The output is the piped to \texttt{pheatmap} which performs the
distance and clustering automatically. The only additional arguements
used here are to change the fontsize and create some breaks in the plot
to highlight the clustering.

There is lots more that \texttt{pheatmap} can do in terms of aesthetics,
so do explore.




\begin{Shaded}
\begin{Highlighting}[]
\NormalTok{dat_scaled }\OperatorTok\StringTok{ }
\StringTok{  }\CommentTok{# Rename the comlums}
\StringTok{  }\NormalTok{magrittr}\OperatorTok{::}\KeywordTok{set_colnames}\NormalTok{(}\KeywordTok{c}\NormalTok{(}\StringTok{"Ctl 1"}\NormalTok{, }\StringTok{"Ctl 2"}\NormalTok{, }\StringTok{"Ctl 3"}\NormalTok{,}
                                    \StringTok{"Trt 1"}\NormalTok{, }\StringTok{"Trt 2"}\NormalTok{, }\StringTok{"Trt 3"}\NormalTok{)) }\OperatorTok\StringTok{ }
\StringTok{  }\CommentTok{# Plot heatmap}
\StringTok{  }\KeywordTok{pheatmap}\NormalTok{(.,}
           \DataTypeTok{fontsize =} \DecValTok{7}\NormalTok{,}
           \DataTypeTok{cutree_rows =} \DecValTok{2}\NormalTok{, }\CommentTok{# Create breaks in heatmap}
           \DataTypeTok{cutree_cols =} \DecValTok{2}\NormalTok{) }\CommentTok{# Create breaks in heatmap}
\end{Highlighting}
\end{Shaded}

\begin{figure}

{\centering \includegraphics[width=0.8\linewidth]{bspr-workshop-2018_files/figure-latex/pheatmap-1} 

}

\caption{Heatmap created using \texttt{pheatmap} with breaks to
highlight clusters.}\label{fig:pheatmap}
\end{figure}

\section{Venn diagram}\label{venn-diagram}

Another common plot used in proteomics is the Venn diagram. For these I
use the \texttt{VennDiagram} package \citep{R-VennDiagram}.

For example if we wanted to compare the protein identifications found in
the control and treatment sets of our data we could compare the protein
accessions found in each condition. To do this we need to transform the
data for example using the following steps:

\begin{enumerate}
\def\labelenumi{\arabic{enumi}.}
\item
\end{enumerate}

\begin{Shaded}
\begin{Highlighting}[]
\CommentTok{# Transform data for Venn diagram to create long table with three columns }
\NormalTok{venn_dat <-}\StringTok{ }\NormalTok{dat }\OperatorTok
\StringTok{  }\CommentTok{# Drop protein description}
\StringTok{  }\KeywordTok{select}\NormalTok{(}\OperatorTok{-}\NormalTok{protein_description) }\OperatorTok\StringTok{ }
\StringTok{  }\CommentTok{# Gather columns according to experiment type to create exp_type }
\StringTok{  }\CommentTok{# and concentration variables. Don't use the protein accession.}
\StringTok{               }\KeywordTok{gather}\NormalTok{(}\DataTypeTok{key =}\NormalTok{ exp_type, }\DataTypeTok{value =}\NormalTok{ concentration, }\OperatorTok{-}\NormalTok{protein_accession)}

\NormalTok{venn_dat}
\end{Highlighting}
\end{Shaded}

\begin{verbatim}
## # A tibble: 46,212 x 3
##    protein_accession  exp_type  concentration
##    <chr>              <chr>             <dbl>
##  1 VATA_HUMAN_P38606  control_1         0.811
##  2 RL35A_HUMAN_P18077 control_1         0.367
##  3 MYH10_HUMAN_P35580 control_1         2.98 
##  4 RHOG_HUMAN_P84095  control_1         0.142
##  5 PSA1_HUMAN_P25786  control_1         1.07 
##  6 PRDX5_HUMAN_P30044 control_1         0.566
##  7 ACLY_HUMAN_P53396  control_1         5.00 
##  8 VDAC2_HUMAN_P45880 control_1         1.35 
##  9 LRC47_HUMAN_Q8N1G4 control_1         0.927
## 10 CH60_HUMAN_P10809  control_1         9.45 
## # ... with 46,202 more rows
\end{verbatim}

\begin{Shaded}
\begin{Highlighting}[]
\NormalTok{venn_cntl_}\DecValTok{1}\NormalTok{ <-}\StringTok{ }\NormalTok{venn_dat }\OperatorTok\StringTok{ }
\StringTok{  }\KeywordTok{filter}\NormalTok{(exp_type }\OperatorTok{==}\StringTok{ "control_1"} \OperatorTok{&}\StringTok{ }\OperatorTok{!}\KeywordTok{is.na}\NormalTok{(concentration)) }\OperatorTok\StringTok{ }
\StringTok{  }\KeywordTok{pull}\NormalTok{(protein_accession)}

\NormalTok{venn_cntl_}\DecValTok{2}\NormalTok{ <-}\StringTok{ }\NormalTok{venn_dat }\OperatorTok\StringTok{ }
\StringTok{  }\KeywordTok{filter}\NormalTok{(exp_type }\OperatorTok{==}\StringTok{ "control_2"} \OperatorTok{&}\StringTok{ }\OperatorTok{!}\KeywordTok{is.na}\NormalTok{(concentration)) }\OperatorTok\StringTok{ }
\StringTok{  }\KeywordTok{pull}\NormalTok{(protein_accession)}

\NormalTok{venn_cntl_}\DecValTok{3}\NormalTok{ <-}\StringTok{ }\NormalTok{venn_dat }\OperatorTok\StringTok{ }
\StringTok{  }\KeywordTok{filter}\NormalTok{(exp_type }\OperatorTok{==}\StringTok{ "control_3"} \OperatorTok{&}\StringTok{ }\OperatorTok{!}\KeywordTok{is.na}\NormalTok{(concentration)) }\OperatorTok\StringTok{ }
\StringTok{  }\KeywordTok{pull}\NormalTok{(protein_accession)}

\NormalTok{venn_list <-}\StringTok{ }\KeywordTok{list}\NormalTok{(}\StringTok{"Control 1"}\NormalTok{ =}\StringTok{ }\NormalTok{venn_cntl_}\DecValTok{1}\NormalTok{,}
                  \StringTok{"Control 2"}\NormalTok{ =}\StringTok{ }\NormalTok{venn_cntl_}\DecValTok{2}\NormalTok{,}
                  \StringTok{"Control 3"}\NormalTok{ =}\StringTok{ }\NormalTok{venn_cntl_}\DecValTok{3}\NormalTok{)}

\NormalTok{futile.logger}\OperatorTok{::}\KeywordTok{flog.threshold}\NormalTok{(futile.logger}\OperatorTok{::}\NormalTok{ERROR, }\DataTypeTok{name =} \StringTok{"VennDiagramLogger"}\NormalTok{)}
\end{Highlighting}
\end{Shaded}

\begin{verbatim}
## NULL
\end{verbatim}

\begin{Shaded}
\begin{Highlighting}[]
\NormalTok{prot_venn <-}\StringTok{ }\KeywordTok{venn.diagram}\NormalTok{(venn_list,}\OtherTok{NULL}\NormalTok{, }
               \CommentTok{#height = 1000,}
               \CommentTok{#width = 1000,}
               \DataTypeTok{col =} \StringTok{"transparent"}\NormalTok{,}
               \DataTypeTok{fill =} \KeywordTok{c}\NormalTok{(}\StringTok{"cornflowerblue"}\NormalTok{, }\StringTok{"green"}\NormalTok{, }\StringTok{"yellow"}\NormalTok{),}
               \DataTypeTok{alpha =} \FloatTok{0.50}\NormalTok{,}
               \DataTypeTok{cex =} \FloatTok{0.8}\NormalTok{,}
               \DataTypeTok{fontfamily =} \StringTok{"sans"}\NormalTok{,}
               \DataTypeTok{fontface =} \StringTok{"bold"}\NormalTok{,}
               \DataTypeTok{cat.col =} \KeywordTok{c}\NormalTok{(}\StringTok{"darkblue"}\NormalTok{, }\StringTok{"darkgreen"}\NormalTok{, }\StringTok{"orange"}\NormalTok{),}
               \DataTypeTok{cat.cex =} \FloatTok{0.8}\NormalTok{,}
               \CommentTok{#cat.pos = 0,}
               \CommentTok{#cat.dist = 0.07,}
               \DataTypeTok{cat.fontfamily =} \StringTok{"sans"}\NormalTok{,}
               \CommentTok{#rotation.degree = 270,}
               \DataTypeTok{margin =} \FloatTok{0.2}\NormalTok{,}
               \DataTypeTok{main =} \StringTok{"Proteins identified in control experiments"}\NormalTok{,}
               \DataTypeTok{main.fontfamily =} \StringTok{"sans"}\NormalTok{,}
               \DataTypeTok{print.mode =} \KeywordTok{c}\NormalTok{(}\StringTok{"raw"}\NormalTok{,}\StringTok{"percent"}\NormalTok{),}
               \DataTypeTok{main.pos =} \KeywordTok{c}\NormalTok{(}\FloatTok{0.5}\NormalTok{,}\FloatTok{0.9}\NormalTok{)}
\NormalTok{  )}

\KeywordTok{grid.arrange}\NormalTok{(}\KeywordTok{gTree}\NormalTok{(}\DataTypeTok{children =}\NormalTok{ prot_venn))}
\end{Highlighting}
\end{Shaded}

\includegraphics{bspr-workshop-2018_files/figure-latex/plot-venn-1.pdf}

\section{Peptide sequence logos}\label{peptide-sequence-logos}

Finally, creating sequence logos from peptides is another common task,
especially if you are doing immunopeptidomics and would like to explore
the fequency of amino acid types at each position in a set of peptide
sequences. The \texttt{ggseqlogo} package enables us to do this
\texttt{ggplot} style \citep{R-ggseqlogo}.

Here using sample data that comes with the ggseqlogo package and
illusrated in the
\href{https://omarwagih.github.io/ggseqlogo/}{ggseqlogo tutorial}.

As with the venn diagram, we need a list to contain our groups of
peptides as they

\begin{Shaded}
\begin{Highlighting}[]
\KeywordTok{data}\NormalTok{(ggseqlogo_sample)}

\KeywordTok{ggseqlogo}\NormalTok{(seqs_aa, }\DataTypeTok{method=}\StringTok{'p'}\NormalTok{)}
\end{Highlighting}
\end{Shaded}

\includegraphics{bspr-workshop-2018_files/figure-latex/ggseqlogo-1.pdf}

\chapter{Going further}\label{going-further}

Here are a few links and suggestions about what else you might like to
do with R.

\section{Exporting figures}\label{exporting-figures}

Exporting figures is best done using the following structure:

\begin{Shaded}
\begin{Highlighting}[]
\CommentTok{# Open up a blank plot file, pdf,jpeg etc.}
\OperatorTok{<}\KeywordTok{plot_function}\NormalTok{(}\StringTok{"file"}\NormalTok{,...)}\OperatorTok{>}
\CommentTok{# Write the plot to the file}
\ErrorTok{<}\NormalTok{plot_object}\OperatorTok{>}
\CommentTok{# Close the file}
\KeywordTok{dev.off}\NormalTok{()}
\end{Highlighting}
\end{Shaded}

For example to export the volcano plot from Figure \ref{fig:nice-vplot}
to a pdf, we do:

\begin{Shaded}
\begin{Highlighting}[]
\CommentTok{# Open up a blank plot file, pdf,jpeg etc.}
\KeywordTok{pdf}\NormalTok{(}\StringTok{"volcano_plot.pdf"}\NormalTok{)}

\CommentTok{# Write the plot to the file}
\NormalTok{dat_tf }\OperatorTok
\StringTok{  }\CommentTok{# Add a threhold for significant observations}
\StringTok{  }\KeywordTok{mutate}\NormalTok{(}\DataTypeTok{threshold =} \KeywordTok{if_else}\NormalTok{(log_fc }\OperatorTok{>=}\StringTok{ }\DecValTok{2} \OperatorTok{&}\StringTok{ }\NormalTok{log_pval }\OperatorTok{>=}\StringTok{ }\FloatTok{1.3} \OperatorTok{|}
\StringTok{                               }\NormalTok{log_fc }\OperatorTok{<=}\StringTok{ }\OperatorTok{-}\DecValTok{2} \OperatorTok{&}\StringTok{ }\NormalTok{log_pval }\OperatorTok{>=}\StringTok{ }\FloatTok{1.3}\NormalTok{,}\StringTok{"A"}\NormalTok{, }\StringTok{"B"}\NormalTok{)) }\OperatorTok
\StringTok{  }\CommentTok{# Plot with points coloured according to the threshold}
\StringTok{  }\KeywordTok{ggplot}\NormalTok{(}\KeywordTok{aes}\NormalTok{(log_fc,log_pval, }\DataTypeTok{colour =}\NormalTok{ threshold)) }\OperatorTok{+}
\StringTok{  }\KeywordTok{geom_point}\NormalTok{(}\DataTypeTok{alpha =} \FloatTok{0.5}\NormalTok{) }\OperatorTok{+}\StringTok{ }\CommentTok{# Alpha sets the transparency of the points}
\StringTok{  }\CommentTok{# Add dotted lines to indicate the threshold, semi-transparent}
\StringTok{  }\KeywordTok{geom_hline}\NormalTok{(}\DataTypeTok{yintercept =} \FloatTok{1.3}\NormalTok{, }\DataTypeTok{linetype =} \DecValTok{2}\NormalTok{, }\DataTypeTok{alpha =} \FloatTok{0.5}\NormalTok{) }\OperatorTok{+}\StringTok{ }
\StringTok{  }\KeywordTok{geom_vline}\NormalTok{(}\DataTypeTok{xintercept =} \DecValTok{2}\NormalTok{, }\DataTypeTok{linetype =} \DecValTok{2}\NormalTok{, }\DataTypeTok{alpha =} \FloatTok{0.5}\NormalTok{) }\OperatorTok{+}
\StringTok{  }\KeywordTok{geom_vline}\NormalTok{(}\DataTypeTok{xintercept =} \OperatorTok{-}\DecValTok{2}\NormalTok{, }\DataTypeTok{linetype =} \DecValTok{2}\NormalTok{, }\DataTypeTok{alpha =} \FloatTok{0.5}\NormalTok{) }\OperatorTok{+}
\StringTok{  }\CommentTok{# Set the colour of the points}
\StringTok{  }\KeywordTok{scale_colour_manual}\NormalTok{(}\DataTypeTok{values =} \KeywordTok{c}\NormalTok{(}\StringTok{"A"}\NormalTok{=}\StringTok{ "red"}\NormalTok{, }\StringTok{"B"}\NormalTok{=}\StringTok{ "black"}\NormalTok{)) }\OperatorTok{+}
\StringTok{  }\KeywordTok{xlab}\NormalTok{(}\StringTok{"log2 fold change"}\NormalTok{) }\OperatorTok{+}\StringTok{ }\KeywordTok{ylab}\NormalTok{(}\StringTok{"-log10 p-value"}\NormalTok{) }\OperatorTok{+}\StringTok{ }\CommentTok{# Relabel the axes}
\StringTok{  }\KeywordTok{theme_minimal}\NormalTok{() }\OperatorTok{+}\StringTok{ }\CommentTok{# Set the theme}
\StringTok{  }\KeywordTok{theme}\NormalTok{(}\DataTypeTok{legend.position=}\StringTok{"none"}\NormalTok{) }\CommentTok{# Hide the legend}

\CommentTok{# Close the file}
\KeywordTok{dev.off}\NormalTok{()}
\end{Highlighting}
\end{Shaded}

If I had saved the plot to an object called \texttt{vplot} I would call
that object instead of making the plot using \texttt{dat\_tf} as shown
here.

\href{https://www.stat.berkeley.edu/classes/s133/saving.html}{Here} is a
general guide to the various formats you can export to.

Alternatively, if you are working in \texttt{ggplot} you can use the
\texttt{ggsave} function as described in
\href{http://r4ds.had.co.nz/graphics-for-communication.html\#saving-your-plots}{R4DS
28.7}.

\section{Exporting data}\label{exporting-data}

There is a
\href{https://cran.r-project.org/doc/manuals/r-release/R-data.html}{full
manual for the import and export of data} in R. However here are few
pointers:

\subsection{Writing to a file}\label{writing-to-a-file}

One of the most portables way to share data is by writing to a csv file.
These files can be opened in many programs. The tidyverse package
contains two functions for csv files, \texttt{write\_csv} and for Excel
\texttt{write\_excel\_csv}. The latter form adds a bit of metadata that
tells Excel about the file encoding. See
\href{http://r4ds.had.co.nz/data-import.html\#writing-to-a-file}{R4DS
writing to a file}.

For example to write a csv file of \texttt{dat\_tf} to a file called
\texttt{04072018\_transformed\_data.csv} to our working directory for
sharing with a colleague using excel, the code is of the form
\texttt{\textless{}function\textgreater{}(\textless{}r-object\textgreater{},"filename")}
like so:

\begin{Shaded}
\begin{Highlighting}[]
\KeywordTok{write_excel_csv}\NormalTok{(dat_tf,}\StringTok{"04072018_transformed_data.csv"}\NormalTok{)}
\end{Highlighting}
\end{Shaded}

Note that the file name is a string and is in quotes.

\subsection{For R}\label{for-r}

If you are exporting data to use yourself in R, the custom \texttt{.rds}
format is a good choice and preserves the R structure.

In the tidyverse \texttt{write\_rds} follows the same structure as
\texttt{write\_csv}.

You can read back in using \texttt{read\_rds}.

\section{Joining the R community}\label{joining-the-r-community}

It's worth joining the \href{https://community.rstudio.com/}{RStudio
Community} and following community members on Twitter such as
\href{https://twitter.com/JennyBryan}{Jenny Bryan},
\href{https://twitter.com/hadleywickham}{Hadley Wickham},
\href{https://twitter.com/xieyihui}{Yihui Xie},
\href{https://twitter.com/dataandme}{Mara Averick},
\href{https://twitter.com/drob}{David Robinson} and
\href{https://twitter.com/juliasilge}{Julia Silge}.

If you can afford \href{https://www.datacamp.com}{DataCamp} then this is
my preferred learning platform.

And if you can't, then \href{https://swirlstats.com/}{swirl} is free.

\section{Communication: creating reports, presentations and
websites}\label{communication-creating-reports-presentations-and-websites}

\href{https://rmarkdown.rstudio.com/lesson-1.html}{R Markdown}
\citep{R-rmarkdown} enables us to do
\href{https://en.wikipedia.org/wiki/Literate_programming}{literate
programming}, saving time as we can create analysis, reports, dashboards
or web apps at the same time as writing code. R Markdown can use
multiple programming languages. See also
\href{http://r4ds.had.co.nz/r-markdown.html}{R4DS R Markdown} and
\href{http://r4ds.had.co.nz/r-markdown-formats.html}{R4DS R Markdown
formats}.

You can use \href{https://bookdown.org/yihui/blogdown/}{blogdown} to
build websites. I created this guide to
\href{http://ab604.github.io/docs/website_bookdown/}{buidling an
academic website with blogdown}.

\subsection{Using bookdown to write a thesis
dissertaion}\label{using-bookdown-to-write-a-thesis-dissertaion}

I used the bookdown package to create these materials \citep{R-bookdown}
and you can use it to write a thesis dissertaion, as detailed very
nicely in this blog by
\href{https://eddjberry.netlify.com/post/writing-your-thesis-with-bookdown/}{Edd
Berry}.

\section{Machine Learning}\label{machine-learning}

If you are interested in machine learning, then
\href{https://tensorflow.rstudio.com/}{TensorFlow} is a good place to
start, for example Leon Eyrich Jessen's
\href{https://tensorflow.rstudio.com/blog/dl-for-cancer-immunotherapy.html}{Deep
Learning for Cancer Immunotherapy} tutorial.

\section{Version control}\label{version-control}

\bibliography{packages.bib,book.bib}


\end{document}
